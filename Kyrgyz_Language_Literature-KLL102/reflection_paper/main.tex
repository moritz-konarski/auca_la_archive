\documentclass[12pt, a4paper, stu]{apa7}

\newcommand{\auth}{Moritz M. Konarski}
\newcommand{\authlast}{Konarski}
\newcommand{\titular}{Reflection Paper on "Kurmanjan Datka: Queen of the
Mountains"}
\newcommand{\shorttitular}{Reflection Paper on Kurmanjan Datka}

\usepackage[english]{babel}
\hypersetup{pdftitle=\titular, pdfauthor=\auth}
\setlength{\parindent}{0.5in}

\title{\titular}
\author{\auth}
\shorttitle{\shorttitular}
\leftheader{\authlast}

\begin{document}

\noindent
Name: \auth{}\\\vspace{-8pt}
\noindent
Date: \today{}\\\vspace{-8pt}
\noindent
Course: Kyrgyz Language \& Literature for Foreign Students, KLL102\\\vspace{-8pt}
\noindent
Instructor: Mukaram Toktogulova\\\vspace{-8pt}

\section{\titular}

\noindent
The movie "Kurmanjan Datka: Queen of the Mountains" (available
\href{https://www.youtube.com/watch?v=I3H86R3zx3E}{here}) details the life and
deeds of Kurmanjan Datka, an important woman in Kyrgyz history. Kurmanjan
Datka's life and the challenges she faces illuminate some of the challenges 
women in 19th century Kyrgyz society faced. In this reflection paper, I will
talk about some of these challenges as they were shown in the movie.\\

The movie begins in 1816 when Kurmanjan is still a little girl. Her father and
mother travel to an old, wise man to receive his blessing for a child. Although
they already have a child (Kurmanjan), they do not see her as having the same
value as a boy. This illustrates how the simple fact of her sex defined her
worth (as lesser than a boy's) in the eyes of her parents and society.\\

The movie jumps a couple of years ahead to when Kurmanjan is a teenager. She
is already married off to a man named Kulseit she does not respect because he 
cannot ride his horse well. By accident, she witnesses a group of men about to 
stone a woman accused of adultery. They are only interrupted because a local 
figure of authority in the Osh region, Alimbek Datka, rides by. 
It turns out that the group of men did not have sufficient evidence to convict 
the woman in the first place. Women did not enjoy a lot of rights 
before male courts that had the power to convict them with very little 
evidence. Like Kurmanjan, many were promised as brides at a young age and 
without their consent.\\

Sometime later, after marrying Kulseit, Kurmanjan runs away because she is
unhappy and he is not a real man. To save herself from her current situation
(she cannot simply divorce Kulseit), she decides to write a proposal letter to 
Alimbek Datka, asking him to marry her. Alimbek Datka comes and marries her. 
This is certainly unorthodox and courageous. With this action, she is breaking 
many social norms to make her life better.\\

After becoming Alimbek's wife, we are shown a crying woman who recently lost
her husband. She is crying because less than a year after her husband's death
his older brother wants to marry her. To maybe do something about this,
Kurmanjan tries to talk to her husband, but because he is a meeting among men,
she is not allowed to talk to him. Here we see a tradition (marrying a deceased
brother's wife) that does not give women a choice in who they marry
(again). Furthermore, a woman is not allowed to participate in a meeting
between men because she is a woman.\\

After Alimbem Datka is beheaded for his betrayal of the Khan of Kokand and war
breaks out between the Alai Kyrgyz and Kokand, the Emir of Bukhara appoints
Kurmanjan as Datka to make peace. Then, much like her late husband, Kurmanjan 
Datka did not succeed in unifying the 40 Kyrgyz tribes. Kurmanjan being named
Datka is significant because that status was generally only given to men. But
because Kurmanjan took over after her husband's death and showed herself as
a capable leader, she was awarded this title.\\

Once the Russian Empire started its expeditions into Kyrgyzstan, they conquered
many tribes and shattered the Kokand Khanate. After Kokand falls, Kurmanjan
Datka becomes the last defender of her people. After fighting that was
devastating for both sides, Kurmanjan Datka brokers a deal with General
Skobelev, the Russian military commander. The Kyrgyz will give up their land,
Kokand will be dissolved, but the Russians promise not to interfere with the
way of life of the Kyrgyz. Here Kurmanjan Datka shows what a capable leader she
is. Even though her decision was controversial, she did what she thought was
best for her people and she stopped the war. These are the kinds of decisions
that only a man would normally make.\\

After General Skobelev is reassigned to the Balkans, the relations with the
Russians deteriorate. Both sides fight and harass each other. One of Kurmanjan
Datka's son's is killed in the fighting and the other one is captured by the
Russians. In an attempt to save her son's life, she offers her own life in his
stead but is rejected. In the end, she watches her son get hanged because she
cannot risk war, regardless of how painful it is to her personally. Kurmanjan
Datka basically lets her son die to save many other lives, a big sacrifice to 
make for anyone.\\

Concluding, I would say that women's situation in 19th century Kyrgyzstan was
not very comfortable. They tended to be subordinate to their husbands and 
traditions. They were also often excluded from political decisions.
Nonetheless, Kurmanjan Datka managed to become a powerful leader and to shape
the history and fate of Kyrgyzstan. For her deeds, she is still remembered and
statues of her are standing in the center of Bishkek.

\end{document}
