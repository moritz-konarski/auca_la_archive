\documentclass[../00_main.tex]{subfiles}

\begin{document}

\section{Nouns}

\subsection{Plural}

\begin{itemize}
    \item plural ending depends on last letter in stem and then on law of
        harmony
    \item if a word ends in a vowel, the plural endings are лар, лор, лер, лөр
    \item if a noun ends in a voiced consonant, the endings are дар, дор, дер, 
        дөр
    \item if a noun ends with an unvoiced consonant, the endings are тар, тор, 
        тер, төр 
    \item this table illustrates this behavior
        \begin{table}[H]
            \center
            \begin{tabular}{| l | l | l |}\hline
                Singular        & Type      & Plural        \\\hline\hline
                Апа             & vowel     & Апалар        \\\hline
                Үй              & vowel     & Үйлөр         \\\hline
                Кыз             & voiced    & Кыздар        \\\hline
                Калем           & voiced    & Калемдер      \\\hline
                Китеп           & unvoiced  & Китептер      \\\hline
                Сүрөт           & unvoiced  & Сүрөттөр      \\\hline
            \end{tabular}
            \caption{Examples of singular and plural forms}
        \end{table}
\end{itemize}

\subsection{Questions}

\subsubsection{Singular}

\begin{itemize}
    \item 2 main question words: Ким? -- Who? and Эмне? -- What?
    \item most often combined with Бул -- This and Aл -- That
    \item examples
        \begin{table}[H]
            \center
            \begin{tabular}{| l | l |}\hline
                Kyrgyz                  & English           \\\hline\hline
                Бул ким?                & Who is this?      \\\hline
                Бул кыз.                & This is a girl.   \\\hline
                Ал ким?                 & Who is that?      \\\hline
                Ал мугалим.             & This is a teacher.\\\hline
                Бул эмне?               & What is this?     \\\hline
                Бул дептер.             & This is a notebook.\\\hline
                Ал эмне?                & What is that?     \\\hline   
                Ал үй.                  & That is a house.  \\\hline
            \end{tabular}
            \caption{Examples of singular noun questions}
        \end{table}
\end{itemize}

\subsubsection{Plural}

\begin{itemize}
    \item for plural questions we use Булар кимдер? -- Who are they? and 
        Булар эмнелер? -- What are they?
    \item examples
        \begin{table}[H]
            \center
            \begin{tabular}{| l | l |}\hline
                Kyrgyz                  & English           \\\hline\hline
                Булар кимдер?           & Who are they?     \\\hline
                Булар студенттер.       & They are students.\\\hline
                Булар мугалимдер.       & They are teachers.\\\hline
            \end{tabular}
            \caption{Examples of plural noun questions}
        \end{table}
\end{itemize}

\subsection{Nominative case}

\subsubsection{Questions}

\begin{itemize}
    \item we use the same questions as in the singular and plural noun section
    \item we have as before Бул / Булар --- This/These and also Тиги / Тигилер 
        --- That/Those
    \item examles: Бул китеп / Булар китептер and Тиги бала / Тигилер балдар
\end{itemize}

\subsubsection{Postpositions}

\begin{itemize}
    \item the nominative case is used with postpositions
    \item here is a list of postpositions
        \begin{table}[H]
            \center
            \begin{tabular}{| l | l |}\hline
                Kyrgyz              & English       \\\hline\hline
                Жөнүндө / тууралуу  & about         \\\hline
                Менен               & by/with/and   \\\hline
                Үчүн                & for/to        \\\hline
                Cыяктуу             & like/as       \\\hline
                Аркылуу             & through       \\\hline
                Боюнча              & according to  \\\hline
            \end{tabular}
            \caption{Examples of postpositions}
        \end{table}
    \item postpositions with pronouns: \ldots менен; the available words are
        \begin{itemize}
            \item Мени
            \item Сени 
            \item Сиз
            \item (ал)- Аны
            \item (бул)-Муну
            \item Биз
            \item Силер
            \item Сиздер
            \item Алар
        \end{itemize}
    \item examples of postpositions in use
        \begin{table}[H]
            \center
            \begin{tabular}{| l | l |}\hline
                Kyrgyz              & English               \\\hline\hline
                Автобус менен барам & I go by bus           \\\hline
                Досум менен барам   & I go with my friend   \\\hline
                Сен жөнүндө         & about you             \\\hline
                Сиз үчүн            & for you               \\\hline
                Биз сыяктуу         & like us               \\\hline
                Сиз аркылуу         & through you           \\\hline
                Китеп боюнча        & according to the book \\\hline
            \end{tabular}
            \caption{Examples of postpositions in use}
        \end{table}
\end{itemize}

\subsection{Genitive case}

\begin{itemize}
    \item the Genitive case is indicated by the ending
    \item the following table illustrates how to find the endings
        \begin{table}[H]
            \center
            \begin{tabular}{| l | l | l |}\hline
                Noun ends with & Ending    & Example   \\\hline\hline
                vowel          & нын       & Баланын, Эненин, Өзүнүн  \\\hline
                Voiced сonsonant& дын      & Кыздын, Үсөндүн, Биздин  \\\hline
                Unvoiced consonant& тын    & Китептин, Азаматтын      \\\hline
            \end{tabular}
            \caption{Findings genitive case endings}
        \end{table}
\end{itemize}

\subsubsection{Multiple possessives}

Here we combine the possessive endings from the possessive pronouns with the
Genitive case to form multiple possessives.
\begin{table}[H]
    \center
    \begin{tabular}{| l | l | l | l | l |}\hline
        N. w/ Gen. e. & N. w/ 3. p. po. e. & Gen. e. & N. w/ 3. p. po.
        & Translation \\\hline\hline
        Аиданын & байкеси & нин & баласы & Аida’s older brother’s son \\\hline
        Айжандын & кызы   & нын & китеби & Aizhan’s daughter’s book   \\\hline
        Тилектин & жеңеси & нин & сумкасы& Tilek’s sister–in–law’s bag\\\hline
    \end{tabular}
    \caption{Forming multiple possessives}
\end{table}

\subsection{Nouns with possessive endings}                                      
                                                                                
\begin{itemize}                                                                 
    \item we take the normal nouns and add the ending --ники, --дыкы, --тыкы    
    \item these words are formed as follows                                     
        \begin{table}[H]                                                        
            \center                                                             
            \begin{tabular}{| l | l | l | l |}\hline                            
                N. ending   & Poss. ending  & Example   & Name Ex.              
                \\\hline\hline                                                  
                vowel       & ныкы      & баланыкы, эженики & Гүлүнүкү,         
                Алияныкы\\\hline                                                
                voiced      & дыкы      & кыздыкы, үйдүкү & Джондуку,           
                Асандыкы\\\hline                                                
                unvoiced    & тыкы      & Мектептики, сабактыкы & Бакыттыкы,    
                Шавхаттыкы\\\hline                                              
            \end{tabular}                                                       
            \caption{Forming nouns with possessive endings}                     
        \end{table}                                                             
\end{itemize}                                                                   

\subsection{Locative case}

Locative case is used to say "where?" and "when?"

\begin{table}[H]
    \center                                                             
    \begin{tabular}{| l | l |}\hline                            
        Kyrgyz          & English       \\\hline\hline
        Where?          &               \\\hline
        Мен мектепте иштейм & I work at the school. \\\hline
        Мен бакта иштейм    & I work at \ldots      \\\hline
        Мен АУЦАда окуйм    & I study at AUCA.      \\\hline
        When?           &               \\\hline
        Дүйшөмбүдө  & On Monday     \\\hline
        Шайшембиде  & On Tuesday    \\\hline
        Жумада      & On Friday     \\\hline
        Ишембиде    & On Saturday   \\\hline
    \end{tabular}                                                       
    \caption{Examples of the locative case}                     
\end{table}                                                             

\subsection{Dative case}

\begin{itemize}
    \item it has 2 meanings: movement directed to a specific location (Кайда?
        Кайсы жакка?); doing something for someone (Кимге? Эмнеге?)
    \item example for case 1: Мен базарга барам. --- I go to the market.
    \item example for case 2: Аидага кат жөнөтөм. --- I will send a letter to
        Aida.
    \item to form words in the dative case:
        \begin{table}[H]
            \center                                                             
            \begin{tabular}{| l | l | l | l |}\hline                            
                N. e. & Dative e. & Ex. & verbs \\\hline\hline
                vowel / voiced   & ГА (ГО, ГЕ, ГӨ) & Шаарга, тоого, үйгө, 
                сизге & БАРАМ, \\ & & & КЕЛЕМ, АЙТАМ \\\hline
                unvoiced & КА (КО, КЕ, КӨ) & Бакка, Болотко, мектепке, Үмөткө
                & КЕЛЕМ, \\ & & & ЖАЗАМ, АЙТАМ \\\hline
                1./2. poss. e. & A (O, Е, Ө) & Балама, тооңо, үйүмө, эжеңе
                & БАРАМ, \\ & & & КЕЛЕМ, АЙТАМ \\\hline
                3. poss. e. & НА (НO, НЕ, НӨ)    & Баласына, эжелерине, үйүнө
                & БАРАМ, \\ & & & КЕЛЕМ, АЙТАМ       \\\hline
            \end{tabular}                                                       
            \caption{Examples of the locative case}                     
        \end{table}                                                             
    \item to go by \ldots: автобус, такси, пароход, ат, машина МЕНЕН БАРАМ, 
        КЕЛЕМ, КЕТЕМ
\end{itemize}

\subsection{Accusative case}

\begin{table}[H]
    \center                                                             
    \begin{tabular}{| l | l |}\hline                            
        Noun        & Ending    \\\hline\hline
        алма        & ны / ни / нү  \\\hline
        нан         & ды / ди / дү  \\\hline
        кант        & ты / ти / тү  \\\hline
    \end{tabular}                                                       
    \caption{Examples of the locative case}                     
\end{table}                                                             

\section{Pronouns}

\subsection{Normal pronouns}

\begin{itemize}
    \item pronouns are used to refer to people
    \item this is a list of the nominative pronouns
        \begin{table}[H]
            \center
            \begin{tabular}{| l | l |}\hline
                Kyrgyz      & English               \\\hline\hline
                Мен         & I                     \\\hline
                Сен         & You (s., informal)    \\\hline
                Сиз         & You (s., formal)      \\\hline
                Ал          & He/She/It             \\\hline
                Биз         & We                    \\\hline
                Силер       & You (pl., informal)   \\\hline
                Сиздер      & You (pl., formal)     \\\hline
                Алар        & They                  \\\hline
            \end{tabular}
            \caption{List of pronouns}
        \end{table}
\end{itemize}

\subsection{Reflexive pronouns}

\begin{itemize}
    \item for example "yourself", "myself", \ldots 
    \item this is a list of the reflexive pronouns
        \begin{table}[H]
            \center
            \begin{tabular}{| l | l |}\hline
                Kyrgyz      & English                   \\\hline\hline
                Өзүм        & myself                    \\\hline
                Өзүң        & yourself (s., informal)   \\\hline
                Өзүңүз      & yourself (s., formal)     \\\hline
                Өзү         & him-, her-, itself        \\\hline
                Өзүбүз      & ourself                   \\\hline
                Өзүңөр      & yourself (pl., informal)  \\\hline
                Өзүңүздөр   & yourself (pl., formal)    \\\hline
                Өздөрү      & themself                  \\\hline
            \end{tabular}
            \caption{List of reflexive pronouns}
        \end{table}
    \item example of usage:
        \begin{dialogue}
            \speak{A} Саламатчылык Муборак. Кандайсын?
            \speak{B} Рахмат, жакшымын. \textbf{Өзүң} кандай?
        \end{dialogue}
\end{itemize}

\subsection{Personal endings I}

\begin{itemize}
    \item personal endings are attached to most words
    \item they indicate who is being talked about
    \item this table shows the personal endings
        \begin{table}[H]
            \center
        \begin{tabular}{| l | l | l |}\hline
            Pronoun     & Ending            & Translation       \\\hline\hline
            Мен         & м(ы,и,у,ү)н       & I                 \\\hline
            Сен         & с(ы,и,у,ү)ң       & You (s, inf)      \\\hline
            Сиз         & с(ы,и,у,ү)з       & You (s, f)        \\\hline
            Ал          & no ending         & He/She/It         \\\hline
            Биз         & б(ы,и,у,ү)з       & We                \\\hline
            Силер       & с(ы,и,у,ү)ң(а,о,э/е,ө)р & You (p, inf)\\\hline
            Сиздер      & с(ы,и,у,ү)зд(а,о,э/е,ө)р& You (p, f)  \\\hline
            Алар        & no ending         & They              \\\hline
        \end{tabular}
        \caption{Personal endings}
        \end{table}
\end{itemize}

\subsection{Personal pronouns II}

\begin{table}[H]
\center
\begin{tabular}{| l || l || l |}\hline
    Positive form       &   Negative form
    & Question form     \\
    stem + P.E.         &   stem эмес + P.E.    & stem + P.E. +     \\
    && б(ы,и,у,ү)? or\\
    && п(ы,и,у,ү)? \\\hline\hline
    pron stem + P.E.    & pron stem эмес + P.E. & pron strem + P.E. + \\
    && question ending  \\\hline
    Мен волонтёр\textbf{мун}. & Биз кыргыз эмес\textbf{пиз}. & Силер
    конок\textbf{суңар}бы? \\
    \textit{I am a volunteer.} & \textit{We are not Kyrgyz.} & \textit{Are you
    guests?}    \\
    &&\\
    Сен саякатчы\textbf{сың}. & Силер бийчи эмес\textbf{сиңер}. & Сиздер
    дос\textbf{суздар}бы?   \\
    \textit{You are a tourist}. & \textit{You are not dancers}. & \textit{Are
    you friends?} \\
    &&\\
    Ал ырчы.    & Алар саякатчы эмес.       & Ал ырчы\textbf{бы}?   \\
    \textit{He is a singer} & \textit{They are not tourists} & \textit{Is he
    a singer?} \\\hline
\end{tabular}
    \caption{More personal endings}
\end{table}

\subsubsection{Examples}

\begin{table}[H]
    \center
\begin{tabular}{| l | l | l |}\hline
    Positive form & Negative form  & Question form
    \\\hline\hline
    Мен студентмин.     & Мен студент эмесмин.      & Мен студентминбы?
    \\\hline
    Биз саякатчыбыз.    & Биз саякатчы эмеспиз.     & Биз саякатчыбызбы?
    \\\hline
    Алар мугалим.       & Алар мугалим эмес.        & Алар мугалимби?
    \\\hline
    Сиз кошунасыз.      & Сиз кошуна эмессиз.       & Сиз кошунасызбы?
    \\\hline
    Сен бир туугансың.  & Сен бир тууган эмессиң.   & Сен бир туугансыңбы?
    \\\hline
    Ал конок.           & Ал конок эмес.            & Ал конокбы?
    \\\hline
    Силер доссуңар.     & Силер дос эмессиңер.      & Силер доссуңарбы?
    \\\hline
    Сиздер бийичисиздер.& Сиздер бийичи эмессиздер. & Сиздер бийичисиздерби?
    \\\hline
    \end{tabular}
    \caption{Examples of positive and negative forms}
\end{table}

\subsection{Possessive pronouns}

\begin{itemize}
    \item to signify ownership, we use possessive pronouns
    \item here is the list of possessive pronouns
        \begin{table}[H]
            \center
            \begin{tabular}{| l | l |}\hline
                Kyrgyz          & English                   \\\hline\hline
                Менин           & My                        \\\hline
                Сиздин          & Your (plural, formal)     \\\hline
                Cенин           & Your (singular, informal) \\\hline
                Анын            & His/her/it                \\\hline
                Биздин          & Our                       \\\hline
                Сиздердин       & Your ( plural formal)     \\\hline
                Силердин        & Your ( plural, informal)  \\\hline
                Алардын         & Their                     \\\hline
            \end{tabular}
            \caption{Possessive pronouns}
        \end{table}
\end{itemize}

\subsection{Pronouns with possessive endings}

\begin{itemize}
    \item we take the normal pronouns and add the ending -ники
    \item here is how these are formed
        \begin{table}[H]
            \center
            \begin{tabular}{| l | l |}\hline
                Kyrgyz          & English           \\\hline\hline
                Меники          & Mine              \\\hline
                Сеники          & Yours             \\\hline
                Сиздики         & Yours             \\\hline
                Аныкы           & His/Her/Its       \\\hline
                Биздики         & Ours              \\\hline
                Силердики       & Yours             \\\hline
                Сиздердики      & Yours             \\\hline
                Алардыкы        & Theirs            \\\hline
            \end{tabular}
            \caption{Pronouns with possessive endings}
        \end{table}
    \item examples of these in sentences
        \begin{table}[H]
            \center
            \begin{tabular}{| l | l |}\hline
                Kyrgyz          & English           \\\hline\hline
                Бул сабак меники. & This is my lesson.              \\\hline
                Бул макала сеники. & This is your article.          \\\hline
                Бул текшерүү иш силердики. & This is their quiz.    \\\hline
                Бул баалар алардыкы. & These are their grades.      \\\hline
            \end{tabular}
            \caption{Examples of pronouns with possessive endings}
        \end{table}
\end{itemize}

\end{document}
