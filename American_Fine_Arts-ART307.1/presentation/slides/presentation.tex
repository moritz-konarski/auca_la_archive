% Presentation for History of American Art for April 28, 2020
% by Moritz M. Konarski
\documentclass[legalpaper,20pt,landscape]{extarticle}

\usepackage[utf8]{inputenc}
\usepackage[english]{babel}
\usepackage[margin=.5in]{geometry}
\usepackage{textcomp}
\usepackage{graphicx}
\graphicspath{{../../graphics/}}
\usepackage{setspace}
\onehalfspace
\usepackage{hyperref}
\hypersetup{
    pdftitle={American Feminist Art},
    pdfauthor={Moritz M. Konarski},
    pdfsubject={American art},
    pdfkeywords={art},
    final
}
\usepackage{enumitem}
\setlist{itemsep=10pt,leftmargin=*}
\usepackage{float}
\pagenumbering{gobble}

\title{American Feminist Art}
\author{Moritz M. Konarski}
\date{April 28, 2020}

\begin{document}
\maketitle
\qquad\begin{minipage}{0.4\textwidth}
\begin{itemize}
    \item questions male dominance in art and society
    \item uses "female crafts" to send a message
    \item not just aesthetic, wants to bring change
    \item currently relevant and evolving
\end{itemize}
\end{minipage}
\begin{minipage}{0.5\textwidth}
\centering
\includegraphics[width=0.5\textwidth]{kruger_your_body}
\end{minipage}
\newpage

\section*{Mary Beth Edelson (1933-- )}

\subsection*{Some Living American Women Artists / Last Supper (1972)}

\begin{minipage}{0.35\textwidth}
\begin{itemize}
\item poster mock-up of \textit{The Last Supper} by Leonardo da Vinci 
    \cite{edelson_supper}
\item pasted over faces of Jesus and his disciples women artists 
\item photos of 82 women artists on this poster \cite{edelson_wikipedia}. 
\item presents female artists, exposes male exclusivity of the art world \cite{edelson_supper}.
\end{itemize}
\end{minipage}\quad
\begin{minipage}{0.65\textwidth}
\includegraphics[width=\textwidth]{edelson_last_supper}
\end{minipage}
\newpage

\section*{Barbara Kruger (1945-- )}

\subsection*{Untitled (I shop therefore I am) (1987)}

\begin{minipage}{.45\textwidth}
\begin{itemize}
    \item print, conceptual work of art -- focused on message \cite{kruger_I_shop}  
    \item \textit{I shop therefore I am} is a spin on Rene Descartes' 
        \textit{I think therefore I am}
    \item comment on materialistic items replacing self-worth of consumers
    \item feminist point of view: male-looking hand holding this message meant 
        for women \cite{kruger_I_shop}
\end{itemize}
\end{minipage}\quad
\begin{minipage}{.50\textwidth}
\includegraphics[width=\textwidth]{kruger_I_shop}
\end{minipage}
\newpage

\subsection*{Untitled (Your Body is a Battleground) (1989)}

\begin{minipage}{.45\textwidth}
\begin{itemize}
    \item protest against laws that attempted to restrict \textit{Roe vs. Wade} 
        that legalized abortions 
    \item made for \textit{March for Women's Lives} in Washington D.C. 
    \item split face could signify ongoing struggle 
        (\cite{kruger_your_body_artstory}, \cite{kruger_your_body_thebroad}).
\end{itemize}
\end{minipage}\quad
\begin{minipage}{.5\textwidth}
\includegraphics[width=\textwidth]{kruger_your_body}
\end{minipage}
\newpage

\subsection*{Magazine Covers --- Rage + Women = Power (1992)}

\begin{minipage}{0.55\textwidth}
\begin{itemize}
    \item Kruger was magazine designer in 70s, these are returns to her roots
    \item right shows cover of \textit{Ms.} magazine from 1992 
        \cite{kruger_rage_women_power}
    \item next slide shows the cover of \textit{W} magazine from 2010 
        (\cite{kruger_about_me_wmagazine}, \cite{kruger_about_me_artstory})
    \item slogans are short and direct and leave room for interpretation
    \item below it is not clear if Kim Kardashian is saying the slogan or if 
        it is a comment on her
\end{itemize}
\end{minipage}\quad
\begin{minipage}{0.5\textwidth}
\includegraphics[height=.95\textheight]{kruger_rage_women_power}
\end{minipage}
\newpage

\subsection*{Magazine Covers --- It's all about me, I mean you, I mean me (2010)}

\begin{figure}[H]
\centering
\includegraphics[height=.9\textheight]{kruger_about_me}
\end{figure}
\newpage

\section*{Guerilla Girls}

\subsection*{Do Women Have to be Naked to Get Into the Met. Museum? (1989)}

\begin{itemize}
    \item from portfolio of 30 images titled \textit{Guerilla Girls Talk Back}
    \item nude woman wearing gorilla mask, based on 
        \textit{La Grande Odalisque} by Ingres 
    \item critiques hypocrisy of Met. Museum -- no problem exhibiting nudes but 
        small number of women artists \cite{guerilla_girls_met}.
\end{itemize}
\begin{figure}[H]
\vspace{-10pt}
\includegraphics[width=.8\textwidth]{guerilla_girls_met_museum}
\centering
\end{figure}
\newpage

\subsection*{The Advantages of Being a Woman Artist (1988)}

\begin{minipage}{.3\textwidth}
\begin{itemize}
    \item poster from the same portfolio as the one above
    \item lists 13 ironic "advantages" of being a woman artist 
    \item highly ironic points highlight irony woman artists' situation
        \cite{guerilla_girls_advantages}
\end{itemize}
\end{minipage}\quad
\begin{minipage}{0.7\textwidth}
\includegraphics[width=\textwidth]{guerilla_girls_woman_artist}
\end{minipage}
\newpage

\section*{Ridykeulous}

\subsection*{The Advantages of Being a Lesbian Woman Artist (2006)}

\begin{minipage}{.3\textwidth}
\begin{itemize}
    \item comment on the Guerilla Girls poster above
    \item crossed things out and re-wrote them in black
    \item changes make the poster ridykeulous
    \item extends critique to the treatment of not heterosexual artists
\end{itemize}
\end{minipage}\quad
\begin{minipage}{0.68\textwidth}
\centering
\includegraphics[height=.88\textheight]{ridykeulous_lesbian_artist}
\end{minipage}
\newpage

\bibliographystyle{siam}
\bibliography{../bibliography}

\end{document}
