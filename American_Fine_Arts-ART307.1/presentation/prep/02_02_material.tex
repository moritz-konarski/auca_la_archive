\documentclass[a4paper,12pt]{article}

\usepackage[utf8]{inputenc}
\usepackage[english]{babel}
\usepackage[final]{hyperref}
\usepackage[margin=1in]{geometry}
\usepackage{graphicx}
\providecommand{\tightlist}{%
  \setlength{\itemsep}{0pt}\setlength{\parskip}{0pt}}
\usepackage{float}

\begin{document}

\setcounter{tocdepth}{4}
\tableofcontents

\section{Material: Art Terms}\label{material-art-terms}

\subsection{Conceptual Art}\label{conceptual-art}

\url{https://www.theartstory.org/movement/conceptual-art/}

\texttt{conceptual\_art\_artstory}: \cite{conceptual_art_artstory}

\begin{itemize}
\tightlist
\item
  ideas are prized over formal or visual components of art works
\item
  mix of tendencies and not super cohesive
\item
  mid 60s to mid 70s they completely rejected standard ideas of art
\item
  articulation of an idea suffices as a work of art
\item
  it was weird to people at first, but like abstract expressionism it
  expanded what art was
\item
  they abandoned beauty, rarity, skill as measures of art
\item
  influenced by Minimalism, readymades
\item
  deeply indebted and based on readymades by Marcel Duchamp --
  idea-based art
\item
  Fluxus (inspiration by everyday things and embrace of change) and
  Minimalism (non-traditional size, materials, production) inspired it
\item
  defiance of the art market where personalities and masterpieces were
  being promoted
\item
  1967 saw the ``manifesto'' by Sol LeWitt: \emph{What the work of art
  looks like isn't too important. It has to look like something if it
  has physical form. No matter what form it may finally have it must
  begin with an idea. It is the process of conception and realization
  with which the artist is concerned.}
\item
  this went against what art was defined as at the time
\item
  some took it so far that they didn't produce physical art at all and
  just left it at the ideas stage
\item
  a good bit of this art was political in nature: commenting on the
  pharmaceutical industry and AIDS epidemic, political oppression
\item
  conceptualism can have a variety of different forms
\item
  Lawrence Weiner in ``Declaration of Intent'': \emph{Art that imposes
  conditions - human or otherwise - on the receiver for its appreciation
  in my eyes constitutes aesthetic fascism}
\end{itemize}

\subsection{Postmodernism}\label{postmodernism}

\url{https://www.tate.org.uk/art/art-terms/p/postmodernism}

\texttt{postmodernism\_tate}: \cite{postmodernism_tate}

\begin{itemize}
\tightlist
\item
  reaction against modernism and its dominance
\item
  associated with skepticism, irony, critique of universal truths and
  objective reality
\item
  first used in 1970
\item
  it is hard to define because it does not follow some kind of theory
\item
  modernism was based on idealism and utopian visions on human life --
  universal principles and truths (religion, science) could ultimately
  explain reality
\item
  thus post-modernism was based in skepticism, suspicion of reason;
  challenged the idea of universal truths -- individual experience is
  more concrete than abstract principles
\item
  \textbf{postmodernism often embraced complex and contradictory layers
  of meaning}
\item
  anti-authoritarian, it refused a single style or definition
\item
  it broke barriers between high culture and everyday culture,
  introduced a sense of anything goes
\item
  it consciously borrows from different styles and styles from the past
  and comments on that too
\end{itemize}

\subsection{Feminist Art}\label{feminist-art}

\url{https://www.moma.org/collection/terms/168}

\texttt{feminist\_art\_moma}: \cite{feminist_art_moma}

\begin{itemize}
\tightlist
\item
  art which seeks to challenge the dominance of men in art and society
\item
  gain recognition, equality for woman artists, question assumptions
  about womanhood
\item
  beginning in the 1960s painting, performance art, and ``women's
  crafts'' were made that aimed to end sexism and oppression
\item
  exposed femininity as masquerade -- set of poses women adopt to
  conform to society
\item
  younger artists also incorporate intersecting concerns about race,
  class, forms of privilege, gender identity and fluidity
\item
  this art style continues to evolve
\end{itemize}

\url{https://www.theartstory.org/movement/feminist-art/}

\texttt{feminist\_art\_artstory}: \cite{feminist_art_artstory}

\begin{itemize}
\tightlist
\item
  emerged in the 1960s amidst general civil rights, anti-war, and queer
  rights movements
\item
  wanted to re-write falsely male-dominated art history
\item
  also wanted to intervene in current culture and transform stereotypes
\item
  created previously non-existent opportunities for women and minority
  artists
\item
  paved the way for identity art of the 1980s
\item
  wanted dialogue through inclusion of womens perspective
\item
  art is not just for aesthetics, but also to incite, to bring change
  towards equality
\item
  creation of alternate venues for women in the arts made exhibitions in
  a prohibitive environment possible
\item
  also worked to change existing policies to promote women
\item
  often used ``female materials'' like textiles or previously seldom
  used media like performance and video -- lacked male-dominated history
  and expanded the definition of fine art
\end{itemize}

\subsection{Photomontage}\label{photomontage}

\url{https://www.tate.org.uk/art/art-terms/p/photomontage}

\texttt{photomontage\_tate}: \cite{photomontage_tate}

\begin{itemize}
\tightlist
\item
  collage constructed from photographs
\item
  often used to express poltical dissent
\item
  first used by dadaists in 1915 in protest against WWI
\item
  then used by surrealists to do their thing
\end{itemize}

\url{https://www.theartstory.org/definition/photomontage/history-and-concepts/}

\texttt{photomontage\_artstory}: \cite{photomontage_artstory}

\begin{itemize}
\tightlist
\item
  was controversial at first because it misrepresented reality but
  people wanted to create things that rivaled fine art
\end{itemize}

\section{Material: Artists}\label{material-artists}

\subsection{Barbara Kruger}\label{barbara-kruger}

\url{https://www.theartstory.org/artist/kruger-barbara/}

\texttt{kruger\_artstory}: \cite{kruger_artstory}

\begin{itemize}
\tightlist
\item
  designer, graphic artist, photographer
\item
  Born: January 26, 1945 - New Jersey
\item
  best known for silkscreen prints that place a concise caption across a
  found photograph
\item
  Reaganomics covered with satire in the 80s
\item
  expanded to installations, video, audio works
\item
  rooted in social, cultural, political critique
\item
  postmodern feminist art, conceptual art, photomontage
\item
  images with short captions allow direct communication with the viewer
\item
  short statements that critique social, societal, political, gender,
  culture\ldots{}
\item
  catchphrases and slick graphical facade
\item
  always contemporary, she sells ideas instead of a product
\item
  appropriates images and adds confrontational phrases
\item
  palette of red, white, and black leaves an impression
\item
  child of legal secretary, father chemical technician
\item
  took art classes in Uni and moved to NYC to take more art classes
\item
  started out with architectural photography, painting, craft, erotic
  imagery
\item
  worked for publications in NYC, lead designer at 22
\item
  was first exhibited in 1973
\item
  took up writing and poetry in the 70s, went back to photography and
  words
\item
  was bases of her later conceptual work
\item
  became more ambitious and critical in her use of text and images,
  started using found images, generally from mass media publications
\item
  style influence by Russian Constructivists, \emph{Alexander Rodchenko}
\end{itemize}

Quotes

\begin{itemize}
\tightlist
\item
  Direct address has been a consistent tactic in my work, regardless of
  the medium that I'm working in.
\item
  I try to deal with the complexities of power and social life, but as
  far as the visual presentation goes I purposely avoid a high degree of
  difficulty. I want people to be drawn into the work.
\item
  Do you know why language manifests itself the way it does in my work?
  It's because I understand short attention spans.
\item
  I basically wasn't cut out for design work because I had difficulty in
  supplying someone else's image of perfection
\item
  I think that art is still a site for resistance \ldots{} I'm trying to
  be affective, to suggest changes, and to resist what I feel are the
  tyrannies of social life on a certain level.
\end{itemize}

\subsubsection{Artworks}\label{artworks}

\paragraph{Untitled (Your Body is a Battleground)
(1989)}\label{untitled-your-body-is-a-battleground-1989}

pic:
\url{https://www.thebroad.org/art/barbara-kruger/untitled-your-body-battleground}

\texttt{kruger\_your\_body\_thebroad}: \cite{kruger_your_body_thebroad}

\begin{itemize}
\tightlist
\item
  protests against antiabortion laws chipping away at 1973 Roe vs.~Wade
\item
  both art and protest
\end{itemize}

\url{https://www.theartstory.org/artist/kruger-barbara/artworks/\#pnt_2}

\texttt{kruger\_your\_body\_artstory}: \cite{kruger_your_body_artstory}

\begin{itemize}
\tightlist
\item
  1989 reproductive rights protest \emph{March for Women's Lives} in
  D.C.
\item
  signature color palette; short, declarative, critical message
\item
  split colors with negative -- struggle, maybe inner struggle
\item
  very direct in message and face staring at you
\end{itemize}

\begin{figure}[H]
\centering
    \includegraphics[width=\textwidth]{../graphics/kruger_your_body.jpg}
\caption{Untitled (Your Body is Your Battleground)}
\end{figure}

\paragraph{Untitled (I Shop Therefore I am)
(1987)}\label{untitled-i-shop-therefore-i-am-1987}

pic: \url{https://publicdelivery.org/barbara-kruger-i-shop/}

\texttt{kruger\_I\_shop}: \cite{kruger_I_shop}

\begin{itemize}
\tightlist
\item
  conceptual work of art
\item
  very small \emph{therefore} -- isn't that obvious and message is still
  sound
\item
  critique of consumerism
\item
  spin on \emph{I think therefore I am} - Rene Descartes
\item
  materialistic items replace self-worth -- shopping defines existence
\item
  focus on having rather than being
\item
  \emph{identity construction through consumption}
\item
  feminist angle: male-looking hand holds up this sign as critique of
  images of women -- women being defined through their appearance
\item
  work can be interpreted from multiple angles
\end{itemize}

\begin{figure}[H]
\centering
\includegraphics[width=\textwidth]{../graphics/kruger_I_shop.jpg}
\caption{Untitiled (I Shop Therefore I am)}
\end{figure}

\paragraph{It's all about me, I mean you, I mean me
(2010)}\label{its-all-about-me-i-mean-you-i-mean-me-2010}

pic:
\url{https://www.wmagazine.com/gallery/kim-kardashian-queen-of-reality-tv-ss/}

\texttt{kruger\_about\_me\_wmagazine}: \cite{kruger_about_me_wmagazine}

\url{https://www.theartstory.org/artist/kruger-barbara/artworks/\#pnt_6}

\texttt{kruger\_about\_me\_artstory}: \cite{kruger_about_me_artstory}

\begin{itemize}
\tightlist
\item
  2010, cover of \emph{W} magazine
\item
  featuring Kim Kardashian
\item
  Futura font, red background
\item
  caption can be critique of the image or exclamation by the subject
\end{itemize}

\begin{figure}[H]
\centering
\includegraphics[width=\textwidth]{../graphics/kruger_about_me.jpg}
\caption{It's all about me, I mean you, I mean me}
\end{figure}

\paragraph{Rage + Women = Power (1992)}\label{rage-women-power-1992}

pic:
\url{https://www.moma.org/collection/works/73568?sov_referrer=art_term\&art_term_id=168}

\texttt{kruger\_rage\_women\_power}: \cite{kruger_rage_women_power}

\begin{itemize}
\tightlist
\item
  cover of \emph{Ms.} magazine
\item
  Kruger was already known at that point
\item
  she investigated the ways in which ideological messages infiltrate
  everyday life
\item
  a return to her roots
\end{itemize}

\begin{figure}[H]
\centering
\includegraphics[width=\textwidth]{../graphics/kruger_rage_women_power.jpg}
\caption{Rage + Women = Power}
\end{figure}

\subsection{Mary Beth Edelson}\label{mary-beth-edelson}

\url{https://www.moma.org/artists/34727}

\texttt{edelson\_moma}: \cite{edelson_moma}

\url{https://en.wikipedia.org/wiki/Mary_Beth_Edelson}

\texttt{edelson\_wikipedia}: \cite{edelson_wikipedia}

\begin{itemize}
\tightlist
\item
  born 1933 as Mary Elizabeth Johnson
\item
  at 13 she organized a support group for people who experienced trauma
\item
  started taking art lessons at this age at the Art Institute of Chicago
\item
  first generation feminist artist
\item
  studied at the Art Institute of Chicago
\item
  got a Master of Fine Arts degree
\item
  taught art at college level
\item
  in 1970s she focused on how women are portrayed in art
\item
  1970s: representations of goddesses -- contrast to patriarchal views
  of women
\item
  she wanted to increase the number of exhibited female artists
\item
  also active in the Civil Right movement
\item
  established the first conference for women in the visual arts in 1968
\item
  leader of the Committee on Diversity and Inclusion and the Women's
  Action Coalition from 1992 to 1994
\end{itemize}

\subsubsection{Artworks}\label{artworks-1}

\paragraph{Some Living American Women Artists / Last Supper
(1972)}\label{some-living-american-women-artists-last-supper-1972}

pic:
\url{https://www.moma.org/collection/works/117141?sov_referrer=art_term\&art_term_id=168}

\texttt{edelson\_supper}: \cite{edelson_supper}

\begin{itemize}
\tightlist
\item
  poster mock-up referencing Leonrado da Vinci's \emph{The Last Supper}
\item
  Jesus and his disciples have their faces covered by Edelson's friends
  or heroes
\item
  she wanted to present the faces of women artists that were seldom seen
  in 1972 while spoofing male exclusivity of patriarchy
\item
  Georgia O'Keeffe is Jesus, everyone else is randomly placed
\item
  Judas is not one of her peers
\end{itemize}

\url{https://en.wikipedia.org/wiki/Mary_Beth_Edelson}

\texttt{edelson\_wikipedia}: \cite{edelson_wikipedia}

\begin{itemize}
\tightlist
\item
  82 women in the image
\end{itemize}

\begin{figure}[H]
\centering
\includegraphics[width=\textwidth]{../graphics/edelson_last_supper.jpg}
\caption{Some Living American Women Artists / Last Supper}
\end{figure}

\subsection{Guerilla Girls}\label{guerilla-girls}

\url{https://www.tate.org.uk/art/artworks/guerrilla-girls-the-advantages-of-being-a-woman-artist-p78796}

\texttt{guerilla\_girls\_advantages}: \cite{guerilla_girls_advantages}

\begin{itemize}
\tightlist
\item
  inception in 1984 in response to a survey at the MOMA that showed that
  of 169 artists less than 10\% were women
\item
  work to expose sexual and racial discrimination in the art world,
  particularly New York
\item
  members wear gorilla masks in public to protect their identities and
  use pseudonyms
\item
  in the 1980s when artwork prices rose steeply, female presence in
  exhibitions diminished dramatically
\item
  call themselves \emph{conscience of the art world}
\item
  target everyone they feel causes or is complicit in the exclusion of
  women and non-white artists
\item
  similar vein to Barbara Kruger who appropriated the visual language of
  advertising
\item
  do lots of work in NYC critical of the museums there
\end{itemize}

\subsubsection{Artworks}\label{artworks-2}

\paragraph{The Advantages Of Being A Woman Artist
(1988)}\label{the-advantages-of-being-a-woman-artist-1988}

pic:
\url{https://www.tate.org.uk/art/artworks/guerrilla-girls-the-advantages-of-being-a-woman-artist-p78796}

\texttt{guerilla\_girls\_advantages}: \cite{guerilla_girls_advantages}

\begin{itemize}
\tightlist
\item
  30 images in portfolio titled \emph{Guerilla Girls Talk Back}
\item
  group of anonymous American female artists
\item
  lists 13 ironic points about being a female artist
\end{itemize}

\begin{figure}[H]
\centering
\includegraphics[width=\textwidth]{../graphics/guerilla_girls_woman_artist.jpg}
\caption{The Advantages Of Being A Woman Artist}
\end{figure}

\paragraph{Do Women Have To Be Naked To Get Into the Met. Museum?
(1989)}\label{do-women-have-to-be-naked-to-get-into-the-met.-museum-1989}

pic:
\url{https://www.tate.org.uk/art/artworks/guerrilla-girls-do-women-have-to-be-naked-to-get-into-the-met-museum-p78793}

\texttt{guerilla\_girls\_met}: \cite{guerilla_girls_met}

\begin{itemize}
\tightlist
\item
  from same portfolio as the previous poster
\item
  reclinig naked women that wears the typical gorilla mask
\item
  image is based on \emph{La Grande Odalisque} (1814) by
  Jean-Auguste-Dominique Ingres
\item
  tried to get this onto advertising space but were rejected based on
  the image
\end{itemize}

\begin{figure}[H]
\centering
\includegraphics[width=\textwidth]{../graphics/guerilla_girls_met_museum.jpg}
\caption{Do Women Have To Be Naked To Get Into the Met. Museum?}
\end{figure}

\subsection{Ridykeulous with Nicole Eisenman, A.L.
Steiner}\label{ridykeulous-with-nicole-eisenman-a.l.-steiner}

\url{https://www.moma.org/collection/works/284636?sov_referrer=art_term\&art_term_id=168}

\texttt{ridykeulous}: \cite{ridykeulous}

\begin{itemize}
\tightlist
\item
  curatorial initiative formed by the Nicole Eisenman, A.L. Steiner in
  2005
\item
  has the goal to encourage the exhibition of queer and feminist art
\item
  uses humor to critique the art world and culture in general
\end{itemize}

\subsubsection{Artworks}\label{artworks-3}

\paragraph{The Advantages of Being a Lesbian Woman Artist
(2006)}\label{the-advantages-of-being-a-lesbian-woman-artist-2006}

pic:
\url{https://www.moma.org/collection/works/284636?sov_referrer=art_term\&art_term_id=168}

\texttt{ridykeulous}: \cite{ridykeulous}

\begin{figure}[H]
\centering
\includegraphics[width=\textwidth]{../graphics/ridykeulous_lesbian_artist.jpg}
\caption{The Advantages of Being a Lesbian Woman Artist}
\end{figure}

\nocite{*}
\bibliographystyle{siam}
\bibliography{bibliography}

\end{document}
