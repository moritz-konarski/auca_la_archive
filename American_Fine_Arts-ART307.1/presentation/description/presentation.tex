%TODO: give my own opinion??
% Presentation for History of American Art for April 28, 2020
% by Moritz M. Konarski
\documentclass[a4paper,12pt]{article}

\usepackage[utf8]{inputenc}
\usepackage[english]{babel}
\usepackage[margin=1in]{geometry}
\usepackage{textcomp}
\usepackage{graphicx}
\graphicspath{{../../graphics/}}
\usepackage{setspace}
\onehalfspace
\usepackage{hyperref}
\hypersetup{
    pdftitle={American Feminist Art - Presentation Description},
    pdfauthor={Moritz M. Konarski},
    pdfsubject={American art},
    pdfkeywords={art},
    final
}
\usepackage{wrapfig}
\usepackage{subfig}

\newcommand{\figref}[1]{\figurename~\ref{#1}}

\title{American Feminist Art\\Presentation Description}
\author{Moritz M. Konarski}
\date{April 28, 2020}

\begin{document}
\maketitle

Feminist art seeks to question the dominance of men in art and society, gain
recognition and equality for women artists, and to question assumptions about
womanhood. Beginning in the 1960s, feminist art used painting, performance art
and "traditionally female" crafts like sowing and weaving to spread its
message. The latter forms of expression were used because they, unlike painting
and sculpture, did not have a male-dominated history associated with them. \newline
This type of art was not just meant to be aesthetic, but also to incite and
bring about change, for example by creating alternate venues for female
artists to exhibit in. Recently, feminist art has also included statements about race, class,
forms of privilege, and gender identity and it is continuously evolving
(\cite{feminist_art_moma}, \cite{feminist_art_artstory}). \newline
Feminist art is connected to conceptual art which prized ideas in art
over the formal execution of the artwork and was often political in its 
messages. For some artists, articulating an idea was enough and they did not
produce physical artworks. The view that the message of an artwork was the
most important thing about it persisted \cite{conceptual_art_artstory}

\section*{Mary Beth Edelson}

Born in 1933, Edelson organized a support group for people who experienced
trauma when she was 13. At the same time, she began taking art lessons at the
Art Institute of Chicago. Edelson is regarded as a first generation
feminist artist, having focused on how women are portrayed in art since the
1970s. She worked to increase the number of exhibited female artists and for
civil rights (\cite{edelson_moma}, \cite{edelson_wikipedia}).

\subsection*{Some Living American Women Artists / Last Supper (1972)}


Edelson's \textit{Last Supper} (from \cite{edelson_supper}) in 
\figref{fig:edelson_last_supper} is a poster mock-up that references
the famous painting \textit{The Last Supper} by Leonardo da Vinci. Edelson
pasted over the faces and heads of Jesus and his disciples with pictures of 
women artists who she placed randomly. All of them are said to be idols and or
artists that she admires. The only deliberate placement was Georgia O'Keeffe as
Jesus, which shows Edelson's admiration for her. All in all, Edelson put the 
photos of 82 women artists on this poster \cite{edelson_wikipedia}. With this 
collage she wanted to present the faces of female artists that were seldomly 
seen in 1972 and expose the male exclusivity of the art world 
\cite{edelson_supper}.
\begin{figure}
\includegraphics[width=\textwidth]{edelson_last_supper}
\centering
\caption{Some Living American Women Artists}
\label{fig:edelson_last_supper}
\end{figure}


\section*{Barbara Kruger}

Barbara Kruger is a designer, graphic artist, and photographer born in 1945 in
New Jersey. She is best known for silkscreen prints that have concise captions
and use found photographs as backgrounds. She started covering
economics and politics satirically in the 80s and continues to critique social, cultural,
and political tidings to this day. The short captions of her works of art make
her communication with the viewer very direct. She uses a palette of red,
white, and black in her artworks to get the viewers attention and to get her
message across \cite{kruger_artstory}.

\subsection*{Untitled (I shop therefore I am) (1987)}

The print in \figref{fig:kruger_I_shop} (from \cite{kruger_I_shop}) is a 
conceptual work of art in the sense that it is more focused on conveying
a message than necessarily being aesthetically pleasing. Still, the striking
red and gray color scheme with white font pops out at the viewer and catches their
attention. The slogan \textit{I shop therefore I am} is a spin on the famous
quote of Rene Descartes \textit{I think therefore I am}. The modification makes
a comment on materialistic 
\begin{wrapfigure}{r}{0.6\linewidth}
\vspace{-10pt}
\includegraphics[width=0.55\textwidth]{kruger_I_shop}
\centering
\caption{Untitled (I shop, therefore I am)}
\label{fig:kruger_I_shop}
\vspace{-8pt}
\end{wrapfigure}
items replacing the self-worth of consumers. This is
an obvious critique of consumerism where shopping can somewhat define one's
existence and the focus is on having rather than being. From a feminist point
of view the artwork can be interpreted as a male-looking hand holding the
message which is meant for women. It tells them that they are defined through
their appearance in a rather typical sexist manner \cite{kruger_I_shop}.
\newline

\subsection*{Untitled (Your Body is a Battleground) (1989)}

\begin{wrapfigure}{l}{0.6\linewidth}
\vspace{-10pt}
\includegraphics[width=0.55\textwidth]{kruger_your_body}
\centering
\caption{Untitled (Your Body is a Battleground)}
\label{fig:kruger_your_body}
\vspace{-10pt}
\end{wrapfigure}
The second work of art by Barbara Kruger in \figref{fig:kruger_your_body} (from
\cite{kruger_your_body_thebroad}) was made in 1989 in connection with protests
against laws that attempted to restrict the 1973 U.S. Supreme Court ruling
\textit{Roe vs. Wade} that legalized abortions in the U.S. Made for the
\textit{March for Women's Lives} in Washington D.C. the artwork shows the same
features as \figref{fig:kruger_I_shop}. The message is very direct, now
even with a face staring at the viewer telling them \textit{Your body is
a battleground} in reference to the battle over reproductive rights. The split 
face could signify the ongoing struggle (\cite{kruger_your_body_artstory}, 
\cite{kruger_your_body_thebroad}).

\subsection*{Magazine Covers}

\figref{fig:kruger_magazine_covers} presents two magazine covers that Kruger
designed. Section (a) shows the cover of \textit{Ms.} magazine from 1992
(from \cite{kruger_rage_women_power}) and (b) shows the cover of \textit{W} magazine 
from 2010 (\cite{kruger_about_me_wmagazine}, \cite{kruger_about_me_artstory}). 
Both covers again demonstrate Kruger's signature style in black, white, and 
red. The slogans are short and direct and leave some room for interpretation.
In (b) it is purposefully not clear if Kim Kardashian is saying the slogan or
if it is a comment on her. These two covers represent a 
return to her roots for Kruger as she started her career as a graphic designer 
at multiple New York City magazines \cite{kruger_artstory}.
\begin{figure}%
\centering
\subfloat[Cover of \textit{Ms.} magazine]{
{\includegraphics[width=0.46\textwidth]{kruger_rage_women_power} }}%
\quad
\subfloat[Cover of \textit{W} magazine]{
{\includegraphics[width=0.46\textwidth]{kruger_about_me} }}%
\caption{Magazine covers}%
\label{fig:kruger_magazine_covers}%
\end{figure}

\section*{Guerilla Girls}

The Guerilla Girls are a group of anonymous female artists that was formed in 
1984 in New York when a survey of the Museum of Modern Art showed that of 169 
exhibited artists less than 10\% were women. Since then they have worked to
expose sexual and racial discrimination in the art world. The Guerilla Girls
wear gorilla masks in public to protect their identities and use pseudonyms
inspired by significant historical female figures. Calling themselves the
\textit{conscience of the art world}, they appropriate the visual language of
advertising to make their point \cite{guerilla_girls_advantages}.

\subsection*{Do Women Have to be Naked to Get Into the Met. Museum? (1989)}

This artwork in \figref{fig:guerilla_girls_met_museum} stems from a portfolio
of 30 images titled \textit{Guerilla Girls Talk Back} and shows text and
a reclining nude woman wearing the typical gorilla mask. This poster is based
on \textit{La Grande Odalisque} by Jean-Auguste-Dominique Ingres and has the
title seen above. In smaller font it says "Less than 5\% of artists in the
Modern Art Sections are women, but 85\% of nudes are female". This poster
critiques the apparent hypocrisy of the Met. Museum that has no problem
exhibiting nudes of women but somehow has a very small number of women artists.
When the Guerilla Girls tried to get this poster displayed on advertisement
boards they were repeatedly rejected based on the nudity of the poster
\cite{guerilla_girls_met}.
\begin{figure}[h]
\includegraphics[width=\textwidth]{guerilla_girls_met_museum}
\centering
\caption{Do Women Have to be Naked to Get Into the Met. Museum?}
\label{fig:guerilla_girls_met_museum}
\end{figure}
\vspace{-20pt}

\subsection*{The Advantages of Being a Woman Artist (1988)}

\begin{figure}[h]
\includegraphics[width=\textwidth]{guerilla_girls_woman_artist}
\centering
\caption{The Advantages of Being a Woman Artist}
\label{fig:guerilla_girls_woman_artist}
\end{figure}
This poster stems from the same portfolio as the one in
\figref{fig:guerilla_girls_woman_artist} and lists 13 very ironic points or
"advantages" of being a woman artist listed beneath the titular headline. Some
of the "advantages" are "Working without the pressure of success" and "Not being
stuck in a tenured teaching position". Obviously these are highly ironic and
supposed to highlight the irony of the situation woman artists find themselves
in \cite{guerilla_girls_advantages}.
\newpage

\section*{Ridykeulous}

Ridykeulous is a curatorial initiative formed by Nicole Eisenman and A.L.
Steiner in 2005 that aims to increase the exhibition of feminist and queer art
and also produces art that uses humor as a critique of the art world and
culture in general \cite{ridykeulous}.

\subsection*{The Advantages of Being a Lesbian Woman Artist (2006)}

This artwork in \figref{fig:ridykeulous_lesbian_artist} is a 
comment on the Guerilla Girls poster from
\figref{fig:guerilla_girls_woman_artist}. Ridykeulous's artwork crossed things
out of the Guerilla Girls poster and re-wrote them with a black
marker. The result is that the items are ridiculous and vulgar \cite{ridykeulous}. 
For example, the title was changed from "The Advantages of Being
a Woman Artist" to "The Advantages of Being a Lesbian Artist". Another point
got changed from "Working without the pressure of success" to "Working without
the pressure of sucking dick". These changes take the already ironic poster up
a notch and making it, as their name suggests, ridykeulous. And while the
poster is very ridiculous, it does extend the critique that was originally
leveled against the art establishment's treatment of women to the treatment of
artists who are not heterosexual.
\begin{figure}[h]
\includegraphics[width=\textwidth]{ridykeulous_lesbian_artist}
\centering
\caption{Advantages of being a woman artist}
\label{fig:ridykeulous_lesbian_artist}
\end{figure}

\bibliographystyle{siam}
\bibliography{../bibliography}

\end{document}
