\documentclass[11pt,a4paper]{article}

\newcommand{\auth}{Moritz M. Konarski}
\newcommand{\authlast}{Konarski}
\newcommand{\titular}{Reflections on the Revolution in France, by Edmund Burke}
\newcommand{\shorttitular}{Reflections on the Revolution in France}

\usepackage[american]{babel}
\usepackage{hyperref}
\hypersetup{pdftitle=\titular, pdfauthor=\auth}

\usepackage[margin=1in]{geometry}

\usepackage{graphicx}
\graphicspath{{../graphics/}}

\usepackage{float}
\usepackage[onehalfspacing]{setspace}
\newcommand{\figref}[1]{\figurename~\ref{#1}}

\usepackage[inline]{enumitem}
\setlist{nosep}

\usepackage[no-math]{fontspec}
\setmainfont
    [Path={/home/moritz/Documents/thesis.git/writing/drafts/00_fonts/}]
    {times.ttf}[
    BoldFont       = timesbd.ttf,
    ItalicFont     = timesi.ttf,
    BoldItalicFont = timesbi.ttf
    ]

\begin{document}

\singlespacing
\noindent
Name: \auth{}\\
\noindent
Date: March 10, 2021\\\vspace{-24pt}

\section*{\titular}\vspace{-4pt}

\onehalfspacing

\paragraph*{Main Idea.}

In this excerpt from 1790, the Anglo-Irish philosopher and one of the founders 
of political conservatism Edmund Burke (1729--1797) comments on the French 
Revolution (1789--1799) and laments the course it took. 
He writes that a state and government ought to be built on ancient 
traditions and rules. Those rules should be inherited from "canonized 
forefathers" (p. 65) and no one should add something unnatural to them; 
creating a government from scratch fills him with disgust. 
Addressing the French people, he tells them they threw away their great
heritage and forgot their honor. Had they built on the existing foundations,
they could have had the best government and state ever. Building on ancient
tradition would have lead to a rich, flourishing, strong, and admirable
country, constitution, monarchy, and people. At the end of this excerpt, he 
states that virtue will bring happiness regardless of social standing, and 
telling a commoner anything else is fiction; civil classes are established for
the benefit of the commoner as much as for the benefit of the lord.\par

From a memory studies perspective, he is advocating for 
remembering the values and virtues of previous generations or even societies.
Those should be used to guide or build current society. In a way, he seems to 
say one should never forget what the forefathers did because their ways
should inform if not govern current decisions.

\paragraph*{Terms.}

\begin{itemize*}
    \item political conservatism
    \item ancient, indisputable laws and liberties
    \item canonized forefathers
    \item low born servile wretches
    \item Maroon slaves
    \item moral equality of mankind
\end{itemize*}

\paragraph*{Select Quotes.}

\begin{itemize}
    \item "The Revolution was made to preserve our ancient, indisputable laws
        and liberties, and that ancient constitution of government which is
        our only security for law and liberty" (p. 65)
    \item "You had all these advantages in your ancient states; but you chose 
        to act as if you had never been moulded into civil society, and had 
        everything to begin anew. You began ill, because you began by despising 
        everything that belonged to you" (p. 66)
    \item "You had a smooth and easy career of felicity and glory laid open to 
        you beyond anything recorded in the history of the world; but you have 
        shown that difficulty is good for men" (p. 67)
\end{itemize}

\paragraph*{Questions.}

\begin{enumerate}
    \item Do you agree with Burke's view of revolution?
    \item If we follow Burke's way of creating government and society, can we 
        ever truly improve either one?
    \item Do you remember the values of your forefathers and do you think of
        them when making decisions?
    \item Looking back, do you agree with Burke's assessment that the French
        Revolution threw away a great opportunity and did not realize its full
        potential?
\end{enumerate}

\end{document}

