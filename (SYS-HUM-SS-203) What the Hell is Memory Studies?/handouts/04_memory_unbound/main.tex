\documentclass[11pt,a4paper]{article}

\newcommand{\auth}{Moritz M. Konarski}
\newcommand{\authlast}{Konarski}
\newcommand{\titular}{Memory Unbound: The Holocaust and the Formation of
Cosmopolitan Memory, by Daniel Levy \& Natan Sznaider}
\newcommand{\shorttitular}{Memory Unbound}

\usepackage[american]{babel}
\usepackage{hyperref}
\hypersetup{pdftitle=\titular, pdfauthor=\auth}

\usepackage[margin=1in]{geometry}

\usepackage[onehalfspacing]{setspace}
\newcommand{\figref}[1]{\figurename~\ref{#1}}

\usepackage[inline]{enumitem}
\setlist{nosep}

\usepackage[no-math]{fontspec}
\setmainfont
    [Path={/home/moritz/Documents/thesis.git/writing/drafts/00_fonts/}]
    {times.ttf}[
    BoldFont       = timesbd.ttf,
    ItalicFont     = timesi.ttf,
    BoldItalicFont = timesbi.ttf
    ]

\begin{document}

\singlespacing
\noindent
Name: \auth{}\\
\noindent
Date: April 12, 2021\\\vspace{-24pt}

\section*{\titular}\vspace{-4pt}

\onehalfspacing

\paragraph*{Main Idea.}

Sociologists Daniel Levy and Natan Sznaider use the term cosmopolitan memory to 
refer to collective memory on a global(ized) scale. This term is needed because 
often collective memory is analyzed only concerning a nation or 
an ethnos. The authors criticize this
practice as ``methodological nationalism" as it ignores the parallels in the
formation of national collective memory and globalized collective memory.
National memory (and the imagined community of a nation) was invented during 
the First
Modernity (transition from agricultural to industrial society and after,
\href{https://en.wikipedia.org/wiki/Second_modernity}{\underline{from Wikipedia}}) and 
provided
representations that could give people a sense of belonging or identity. In
this process, local cultures were changed and transformed by the national
culture. They argue now in the Second Modernity (transformation of industrial
society into an information society) cosmopolitan memory is the result of
the same processes where national memories are being transformed by
transnational ones. 
What sets the transitions in the First and Second Modernity apart is that in the
former, heroic myths were used to set oneself apart from others and narratives
were about ``acting perpetrators". In the latter, self-critical awareness of the
national past is commonplace and injustices committed by a nation are the focus. 
They call this the narrative of the ``non-acting" victim--the history and 
memories of the ``Other" are being recognized. This also leads to a loss of
distinction between the history of the perpetrator and that of the victim. As
a result, a shared past remains. The main example the authors give for this
development is the Holocaust. Over time, (and because most people who
experienced it are now dead) it became less about remembering the atrocities and
more about each group dealing with their legacy and recognizing each other's
coping process.\vspace{-10pt}

\paragraph*{Terms.}

\begin{itemize*}
    \item transnational memory
    \item cosmopolitan memory
    \item watershed in European history
    \item globalization
    \item imagined community
    \item traditional and exemplary narratives
    \item the ``Other"
    \item First and Second Modernity
\end{itemize*}\vspace{-10pt}

\paragraph*{Select Quotes.}

\begin{itemize}
    \item ``[S]hared memories of the Holocaust [\dots] provide the foundations
        for a new cosmopolitan memory, a memory transcending ethnic and
        national boundaries." (p. 465; pdf 479)
    \item ``To say that nations are the only possible containers of true history 
        is a breathtakingly unhistorical assertion." (p. 466; pdf 480)
    \item ``Rather than privileging one form of memory over the other, it seems 
        more fruitful to identify the different historical and sociological
        conditions of memory cultures." (p. 466; pdf 480)
    \item ``The concept of ``cosmopolitan memory" corresponds to the globalized 
        horizon of experiences in Second Modernity." (p. 467; pdf 481)
\end{itemize}\vspace{-10pt}

\paragraph*{Questions.}

\begin{enumerate}
    \item Regarding your sense of belonging, do you feel like you belong 
        exclusively to your nation or something more transnational? 
    \item Thinking about your country, do you think people have a self-critical
        awareness of its past or are heroic narratives the norm?
    \item In your culture (or one you are familiar with), which group is the
        ``Other" (or victim) that is or should be recognized?
\end{enumerate}

\end{document}

