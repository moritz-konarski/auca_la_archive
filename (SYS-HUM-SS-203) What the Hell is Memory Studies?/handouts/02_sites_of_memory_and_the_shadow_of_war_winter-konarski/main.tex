\documentclass[11pt,a4paper]{article}

\newcommand{\auth}{Moritz M. Konarski}
\newcommand{\authlast}{Konarski}
\newcommand{\titular}{Sites of Memory and the Shadow of War, by Jay Winter}
\newcommand{\shorttitular}{Sites of Memory and the Shadow of War}

\usepackage[american]{babel}
\usepackage{hyperref}
\hypersetup{pdftitle=\titular, pdfauthor=\auth}

\usepackage[margin=1in]{geometry}

\usepackage{graphicx}
\graphicspath{{../graphics/}}

\usepackage{float}
\usepackage[onehalfspacing]{setspace}
\newcommand{\figref}[1]{\figurename~\ref{#1}}

\usepackage[inline]{enumitem}
\setlist{nosep}

\usepackage[no-math]{fontspec}
\setmainfont
    [Path={/home/moritz/Documents/thesis.git/drafts/00_fonts/}]
    {times.ttf}[
    BoldFont       = timesbd.ttf,
    ItalicFont     = timesi.ttf,
    BoldItalicFont = timesbi.ttf
    ]

\begin{document}

\singlespacing
\noindent
Name: \auth{}\\
\noindent
Date: \today{}\\\vspace{-24pt}

\section*{\titular}\vspace{-4pt}

\onehalfspacing

\paragraph*{Main Idea.}

Historian Jay Winter uses his knowledge of World War I (WWI) to examine the way
it influenced sites of memory. He defines sites of memory as physical places 
where people go to publicly remember a defining aspect of their shared past
(WWI and WWII in this case). Winter states that: 
"Public commemoration is both irresistible and unsustainable. Constructing 
sites of memory is a universal social act, and yet these very sites are as 
transitory as are the groups of people who create and sustain them" (p.
73). After WWI, people wanted a place to remember the dead---even just by their 
names. These sites of memory often looked for meaning in what had 
happened and displayed power, national identity, or invented tradition. 
Realizing this goal entails artistic, representational, and monetary challenges
(the Holocaust is particularly difficult). 
To have meaning, a site needs people to have a connection to it through a family 
memory---generally a death. If these connections disappear, the site soon follows.

\paragraph*{Terms.}

\begin{itemize*}
    \item sites of second-order memory
    \item World War 1, The Great War
    \item commemorative process
    \item collective shared knowledge
    \item institutionalization and routinization
    \item Armistice Day
    \item Holocaust
    \item Shoah 
    \item fashioning of narratives about the past
    \item the invention of tradition
    \item utilitarian memorials
    \item business of remembrance
    \item semiotics
    \item medievalism
    \item "lost generation"
    \item family transmission of narratives
\end{itemize*}

\paragraph*{Select Quotes.}

\begin{itemize}
    \item "Twentieth-century warfare democratized bereavement. Previously 
        armies were composed of mercenaries, volunteers and professionals. 
        After 1914, Everyman went to war." (p. 68)
    \item "[T]he extreme character of the Second World War challenged the 
        capacity of art---any art---to express a sense of loss when it is 
        linked to genocidal murder or thermonuclear destruction." (p. 70)
    \item "Commemorative ritual survives when it is inscribed within the 
        rhythms of community and, in particular, family life." (p. 71)
\end{itemize}

\paragraph*{Questions.}

\begin{enumerate}
    \item Do you have a connection to the events of WWI or WWII through a family
        memory? If you do, does it influence the way you think about those
        events? Are there other events (from before your birth) that you have
        such a connection to?
    \item Winter states that it is the natural course of events that sites of
        memory eventually fade away or change their meaning. Are there 
        sites where that cannot be allowed to happen? What about Auschwitz 
        or Hiroshima?
    \item Do you think remembering events through second-order memories and
        specifically crafted sites is a good idea? Do you think this
        practice is dangerous?
    \item How should sites of memory that society no longer deems
        worthy be treated? Should we force ourselves to forget certain
        sites of memory?
\end{enumerate}

\end{document}

