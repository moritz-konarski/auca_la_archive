\documentclass[12pt, a4paper, stu]{apa7}

\newcommand{\auth}{Moritz M. Konarski}
\newcommand{\authlast}{Konarski}
\newcommand{\titular}{South Africa's Truth and Reconciliation
Commission:\\Effective Transition or Blanket Amnesty?}
\newcommand{\shorttitular}{South Africa's Truth and Reconciliation Commission}

\usepackage[no-math]{fontspec}
\setmainfont
[Path={/home/moritz/Documents/thesis.git/writing/drafts/00_fonts/}]
    {times.ttf}[
    BoldFont       = timesbd.ttf,
    ItalicFont     = timesi.ttf,
    BoldItalicFont = timesbi.ttf
]

\usepackage[american]{babel}
\usepackage{csquotes}
\usepackage[style=apa]{biblatex}
\addbibresource{../bib.bib}

\hypersetup{
   colorlinks=true,
   allcolors=black,
   urlcolor=blue,
   urlbordercolor=blue,
   final,
   pdftitle={\titular{}},
   pdfauthor={\auth{}},
   pdfsubject={Memory Studies, Relection Paper}
}

\setlength{\parindent}{0.5in}

\title{\titular}
\author{\auth}
\shorttitle{\shorttitular}
\leftheader{\authlast}

\begin{document}

\noindent
Assignment 3 Option c)\\\vspace{-8pt}
\noindent
Name: \auth{}\\\vspace{-8pt}
\noindent
Date: \today{}\\\vspace{-8pt}
\noindent
Course: What in the Hell is Memory Studies? \\\vspace{-8pt}
\noindent
Instructors: Prof. Clyde R. Forsberg \& Hatice E. Mescioglu\\\vspace{-8pt}

\section{\titular}

\noindent
In the year 1990, South Africa started its transition to a democratic state
after 40 years of racial oppression.
As part of this process, the crimes of the old regime
needed to be acknowledged and legally dealt with--a process called
\textit{transitional justice}. Transitional justice is the legal reckoning with
the transition from a criminal, authoritarian state to a democratic one
\parencite{langenohl2008}. This process can yield a ``clear identification of
victims and perpetrators, the validity of which is emphasized by legal
sanctions" \parencite[p. 166]{langenohl2008}; it can make grappling with the
criminal past easier. Transitional justice struggles with the fact
that many authoritarian \textit{macro-crimes}, as \citeauthor{langenohl2008}
calls them, are not committed by individuals, but supported or profited from by
a large section of the population--prosecuting individuals for these crimes is
very difficult. This issue makes transitional justice alone an insufficient tool 
for the creation of
a democratic state \parencite[p. 168]{langenohl2008}. A Truth and Reconciliation
Commission (TRC) is one proposed answer to this
``contradiction between the political imperative to integrate a society in
transition--victims, perpetrators, bystanders, and profiteers--and the ethical,
social, and juridical imperatives to do justice to victims and to indict
perpetrators" \parencite[p. 168]{langenohl2008}. A TRC would not prosecute
offenders, but try to establish a common public record (or ``truth") of what
happened, and hopefully help bring about reconciliation between
the opposing parties \parencite[p. 168]{langenohl2008}.\par

The most widely know TRC is the South African Truth and Reconciliation
Commission, which, even though it was not the first, has influenced many
successive TRCs \parencite{vora2004,langenohl2008}. The South African TRC
(hereinafter ``the TRC") was 
established to nonviolently address the legacy of \textit{apartheid} in South 
Africa.
Apartheid was a system of racial
segregation and oppression based on white supremacy of the Afrikaners
(descendants of Dutch settlers) and English South Africans (descendants of
English settlers) over the Black majority of the South African population and 
the Asian and ``Coloured" (mixed-race) minorities \parencite[p.
304]{vora2004}. Although racial inequality in South Africa goes back hundreds of
years, apartheid was a special, institutionalized form of it introduced in 1948
by the Afrikaner Nationalist Party. It systematically oppressed Black South
Africans by not allowing them to vote, banning them from certain jobs, evicting
them from cities, and denying their right to vote, and more 
\parencite[p. 305]{vora2004}. When apartheid ended in 1990, the oppressors and
the oppressed negotiated an interim constitution that should enable free and
open elections. The White elite only allowed this if they were to be
granted amnesty for human rights abuses they committed during apartheid
\parencite{vora2004,tutu2019}. The resulting constitution included a clause
granting said amnesty and this laid the groundwork for the creation of the TRC in
1995. This interim constitution represented a compromise between the demands of the
political right and the security forces, who wanted complete amnesty for their
abuses, and the liberation movements and activists, who wished for trials
similar to those in Nuremberg in 1945. As chairman of the TRC, Nelson Mandela
appointed Archbishop and Nobel laureate Desmond Tutu \parencite{tutu2019} (he
wrote the article cited here). The TRC's explicit purpose was to uncover and 
document
the human rights abuses committed under apartheid, give all ethnic minorities
a place to speak and tell their stories, and to create a reparations policy for
the government. Furthermore, it had the power to grant amnesty to perpetrators
of human rights abuses if they publicly admitted their crimes and thus faced
their victims \parencite{tutu2019}. Another goal of the TRC was, as the name
suggests, to reconcile the divided South African nation and thus engage in
nation building. It was believed that the past should not be forgotten, but
that it needed to be publicized and addressed \parencite[p. 306]{vora2004}.
During public hearings across South Africa, the TRC recorded over 22,000
testimonies of gross violations of human rights. Those are ``defined in the Act
as torture, killings, disappearances and abductions, and severe ill treatment
suffered at the hands of the apartheid state" \parencite{tutu2019}. People who
were wronged by the liberation movements were also heard.\par

The complex issues the TRC was tasked with solving make it difficult to assess
the TRC's effectiveness. \textcite{tutu2019} names the following shortcomings
of the commission: 
\begin{itemize}
    \item top military and political leaders did not participate in the process 
        and did not ask for amnesty;
    \item liberation movement fighters claimed that they fought a ``just war" 
        and thus had no need for amnesty;
    \item the economy of the apartheid regime was not scrutinized enough;
    \item the South African government was slow to implement the TRCs 
        suggestions; and
    \item the prosecution of those who did not apply for amnesty was slow or
        nonexistent.
\end{itemize}
In \citeauthor{tutu2019}'s view, this failure to prosecute made victims
distrust the process and let many people escape justice. Nonetheless, his
overall assessment of the TRC is positive, but this should be seen critically 
as he was its chairman. \textcite{tutu2019} sees the public participation in the 
process, 
the hearing of victims \textit{and} perpetrators, and the use of amnesty as a 
compromise as the main achievements of the TRC. It started a process of 
reconciliation in South Africa and can, \citeauthor{tutu2019} hopes, serve as a 
basis for TRCs in other countries.\par

To investigate the effects and perceptions of the TRC in South Africa,
\textcite{gibson2005} performed a representative national survey. The survey is
categorized by the South African ``races" because opinions are highly
correlated with and divided by those categories. Concerning their opinion of the 
TRC in general, about 75\% of
Black South Africans and only 36\% of White South Africans approve of the
commission's work. The South African population as a whole judges that the TRC
uncovered the truth well, and their work on compensating victims is most
criticized \parencite[p. 347]{gibson2005}. Granting amnesty to perpetrators is
seen as unfair by a majority of South Africans; it is considered especially
unfair to the people who died during apartheid's struggles. \textcite[pp.
349--350]{gibson2005} says that while people judge amnesty as being unfair, it
may have been a ``necessary evil" and it does not detract from peoples' view of
the TRC. He furthermore found a ``significant positive relationship [\ldots]
between support for amnesty and confidence in the South African legal system"
\parencite[pp. 350--351]{gibson2005}. South Africans assign blame pretty
evenly to the groups and institutions involved in apartheid \parencite[p.
352]{gibson2005} and see it as a crime against humanity \parencite[p.
354]{gibson2005}. In spite of that, between 35\% and 50\% of the population 
think that the ideas of apartheid were good, and 24\% to 39\% think that the 
people defending apartheid were being just \parencite[p. 354]{gibson2005}. 
\textcite[p. 355]{gibson2005} evaluates
this positively because the TRC process made people ``see the past in equivocal
terms, not as a struggle between absolute good and infinite evil," which can
pave the way for reconciliation.\par

In evaluating the TRC, I would like to use the subtitle of this 
paper--``Effective Transition or Blanket Amnesty?"--as a guide. In my opinion, 
these
two elements are most important when it comes to the South African TRC. The
first one, effective transition, is the facilitation of a peaceful transition
from apartheid to a democratic state. The second one, blanket amnesty, is
a provocative statement about transitional justice and whether justice has been
done or not. The TRC has, in my opinion, succeeded in bringing about a peaceful
and effective transition from the authoritarian apartheid regime to a more
democratic and inclusive state. As mentioned, before amnesty was written into
the interim constitution, a peaceful transition of power was not guaranteed.
The amnesty stipulation guaranteed this transition and also paved the way for
the TRC to be established. The TRC then worked with the goal of reconciliation
in mind, not prosecuting and condemning people, but giving them a chance to
speak and even ask for forgiveness, which further facilitated a peaceful
transition of power. Had they cracked down on the perpetrators, the transition
may not have been peaceful.\par

Assessing if justice was done is more difficult. During transitions of
power, the identification of victims and perpetrators is important, and the TRC
did that through 22,000 victims' statements and public broadcasting
of its hearings. These records also present a very large collection of
experiences of during apartheid. These records can be used to shape
the perception of the events and in turn shape how they are collectively
remembered. But justice is more than just the documentation of abuses, it 
includes the sanctioning of actions. This is where the TRC fell short, but that
is not just its fault. Not getting high-ranking government and
military officials to participate in the TRC certainly lessened its
effectiveness because the people who orchestrated apartheid were not part of
the reconciliation process and did not take responsibility as part of the
process.
Additionally, not examining the economic dimension of apartheid in great detail
ignores a vital part of the system. The TRC's use of amnesty is also seen as
unfair by the South African population \parencite[p. 349]{gibson2005}.
Criticizing these shortcomings, it is important to keep in mind that the TRC
was not supposed to prosecute people, it was not a court or a trial. The word
\textit{justice} is not in its name, but reconciliation and truth are. Thus
I think it is not quite fair to judge the TRC's failure to bring justice
because
that was never a core part of its objectives. What it should be judged on is its
ability to bring out the truth and to facilitate reconciliation. The recording
of 22,000 victim statements does not embody the whole truth, but it represents
the experiences of ordinary people that, without the TRC, would most likely
have stayed their personal experiences, never to be shared with a wider
audience. The survey by \citeauthor{gibson2005} suggests that the TRC was
successful in establishing the main facts of apartheid as truth in all racial
groups in South Africa. Concerning reconciliation, the TRC had a very difficult
task ahead of it. South Africa was very divided and the wounds causing that
division were very deep. \citeauthor{gibson2005}'s research suggests that in
\citeyear{gibson2005}, South Africa was still divided across racial lines, but
that these divisions were not absolute and that a sizable portion of each
racial group agrees with another racial groups, meaning they are not completely
split along racial lines. This can be interpreted as a step toward
reconciliation if it is assumed that these divisions were much stronger in the
past. A further assumption is that the TRC contributed to this softening of
divisions. \textcite{tutu2019} and \textcite{gibson2005} suggest that this is
the case. This process is still underway which, I think, is characterized by 
the uncertain opinions that South Africans have towards the
defenders and ideas of apartheid \parencite[p. 354]{gibson2005}.\par

Finally and to conclude, I think that the TRC successfully facilitated
a peaceful transition of power in South Africa. This is a very important
achievement considering the volatile state South Africa was in after the end of
apartheid. When it comes to doing justice, I do not think it brought a lot of
justice to South Africa, but this should not be counted against the TRC too
much as this was never its main goal. Despite the shortcomings that were
discussed above, I think that, overall, the TRC was a success. It compiled
a record of peoples' experiences, allowed victims and perpetrators alike to tell
their stories, and granted amnesty to those who showed remorse and apologized.
To answer the question ``Effective Transition or Blanket Amnesty?," the TRC
facilitated an effective transition, but not through blanket amnesty (the TRC 
only granted 1,500 amnesty requests out of the 7,000 they
received \parencite{tutu2019}).\par

From a memory studies perspective, the TRC is an
interesting attempt at creating a collective memory of apartheid. It attempted
to take individual peoples' experiences and compile them into a large, shared
history for all of South Africa. The focus on reconciliation and the open
nature of the TRC softened the division represented in this history (compared
to ruthless persecution potentially hardening those divisions). Thus, the 
collective memory generated by the TRC would consist of South African 
experiences and come ``from within"--focusing on a shared history that South 
Africans experienced, overcame, and that they can use as foundations to build 
a better country.

\printbibliography{}

\end{document}
