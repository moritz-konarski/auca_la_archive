\documentclass[12pt, a4paper, stu]{apa7}

\newcommand{\auth}{Moritz M. Konarski}
\newcommand{\authlast}{Konarski}
\newcommand{\titular}{James E. Young's Analysis of the Process of Memorialization}
\newcommand{\shorttitular}{James E. Young's Process of Memorialization}

\usepackage[no-math]{fontspec}
\setmainfont
[Path={/home/moritz/Documents/thesis.git/writing/drafts/00_fonts/}]
    {times.ttf}[
    BoldFont       = timesbd.ttf,
    ItalicFont     = timesi.ttf,
    BoldItalicFont = timesbi.ttf
]

\usepackage[american]{babel}
\usepackage{csquotes}
\usepackage[style=apa]{biblatex}
\addbibresource{../bib.bib}  

\hypersetup{
   colorlinks=true,
   allcolors=black,
   urlcolor=blue,
   urlbordercolor=blue,
   final,
   pdftitle={\titular{}},
   pdfauthor={\auth{}},
   pdfsubject={Memory Studies, Relection Paper}
}

\setlength{\parindent}{0.5in}

\title{\titular}
\author{\auth}
\shorttitle{\shorttitular}
\leftheader{\authlast}

\begin{document}

\noindent
Assignment 2 Option a)\\\vspace{-8pt}
\noindent
Name: \auth{}\\\vspace{-8pt}
\noindent
Date: \today{}\\\vspace{-8pt}
\noindent
Course: What in the Hell is Memory Studies? \\\vspace{-8pt}
\noindent
Instructors: Prof. Clyde R. Forsberg \& Hatice E. Mescioglu\\\vspace{-8pt}

\section{\titular}

\noindent
In the study of memory and memorials, few academics are as prominent as James 
E. Young, a Professor Emeritus at the University of Massachusetts,
Amherst. There, he taught English and Judaic \& Near Easter Studies and founded
the Institute for Holocaust, Genocide, and Memory Studies. As a visiting
professor, he taught at Harvard University and Princeton University among
others \parencite{zotero-381}. Young has written multiple books on the 
Holocaust and the nature of memory and he served on the jury for the German 
\textcite{2020a} in Berlin, the 
U.S.-American \textcite{zotero-397} in New York 
City \parencite{zotero-381}, and most recently as a consultant for the Norwegian 
government following the Utøya mass shooting in 2011 \parencite{ehrenpreis2018}. 
Additionally, he curated an art exhibition in 1994-1995 titled ``The Art of 
Memory: Holocaust Memorials in History" \parencite{zotero-381} which focused on 
memorials of the Holocaust and on the nature of memorials themselves.
He possesses a plethora of both theoretical and practical experience 
concerning memorials and memorization.\par

In the chapter ``The Texture of Memory: Holocaust Memorials in History" from
the book \textit{Cultural Memory Studies} \parencite{young2008}, Young 
discusses the 
process of memorial creation, difficulties in representation, and problems with
traditional monuments with a special focus on the Holocaust. He writes that
memorials are dependent on their location, their creator, the history and life 
of their state and community, and even the current memory-artistic movements.
Any Holocaust memorial has multifaceted meaning and its meanings may change
over time, regardless of the original creator's intentions. While each country
faces challenges remembering the Holocaust, Germany faces them in their most
extreme. Redeeming the Holocaust through art, making shame part of a national
memorial, and how a perpetrator can remember its victims are some of the
issues that German memorial artists must face. Memorials that address these 
issues would
require what Young calls a ``breach in conventional ``memorial code""
\parencite[p. 359]{young2008}. Recognizing this conundrum, German artists
turned to ``counter-monuments," which do not make the visitor a passive element
of the memorial, do not attempt to replace the actual past, and often
incorporate the above-mentioned issues into the monuments themselves. This
shift resulted in memorials that fittingly commemorated events through
analogies, for example commemorating the destruction of a people with the 
destruction of a building. In contrast to normal monuments, counter-monuments 
do not attempting to encapsulate or seal off the contended and evolving memory
that they represent. The artists 
also recognized that the process of memorial creation is a central 
part of the memorial itself. Toward the end of the chapter,
Young discusses the interaction between art-historical analysis and Holocaust 
memorials. He criticizes that such analysis
often ignores the fact that memorials are made for the public (i.e. that mass
appeal can be a good thing) and that critics apply the same criteria to
memorial analysis that they apply to ``high art". In doing that, they ignore 
the different purposes these art forms serve, and thus their analysis is not
correct. Furthermore,
the author advocates for the use of memory studies methods in memorial art 
critique by asking ``how memorial representations of history may finally 
weave themselves into the course of ongoing events" \parencite[p. 362]{young2008}. 
Central to Young's approach is the role of the monument in the present
moment and how ``[it] suggests itself as a basis for political and social 
action" \parencite[p. 362]{young2008}. Studying the actions that monuments
inspire and being aware of the process of their creation will produce a more
complete picture of their influence and remind us of their changing meaning and
dependence on visitors to give them meaning in the first place. Young argues
against setting memory and memorials in stone and hopes that by incorporating 
the memorial process into the finished memorial we can save it from becoming
disjointed from its process of creation. To summarize, Young argues for an art 
history and memorial critique that considers how a memorial shapes our 
understanding of an era, how this understanding, in turn, influences our view of 
the present, and how this process can ultimately lead to action.\par

In my opinion, Young's argument is very though out and reasonable. In 
considering not just
the artistic aspect of the monument, but its meaning, its influence on peoples'
actions, and process of its creation we will be able to judge or
analyze it more completely. This presupposes that we accept that a monument is
more than just the physical object, i.e. that it has a memory dimension and 
that it can influence visitors' opinions and views. The focus on the memorial 
process as an addition to the finished memorial enables a better understanding
of the meaning of the memorial and even its artistic form. Young talks about
this in detail in a talk titled ``The Stages of Memory: Reflections on 
Memorialisation and Global Commemoration" which is available on YouTube
\parencite{britishassociationforholocauststudies2020}. Starting at 46:50, he
talks about the process of creating the \textcite{zotero-397} which he was
involved in as a juror. For him, the memorial process started
with the first news reports of the terror attacks, continued with spontaneous
candlelight vigils and the posting of missing person flyers all over New York 
City. On the day of the six month anniversary of the attacks, two beams of 
light were
shone into the sky, signifying the voids left behind by the towers. For the
first anniversary, small pools of water were built in the center of the now
cleared foundations, and the survivors and families of the deceased laid flowers
onto the pools and read out the names of all the victims. The actual memorial
design process consisted of choosing one design out of 5201 submitted designs.
The winning design focused on two large, deep pools with waterfalls 
representing both the voids left behind by the towers and their fall. The
pools are surrounded by trees that block out the city and invite people
to perambulate, as Young puts it. The idea to make the process of memorial creation part
of the memorial itself seems the only reasonable thing to do, as for the people with
the closest ties to the commemorated event, the creation of a monument itself
is part of the process or working through and understanding an event.\par

The above mentioned perambulation--leaving the visitor to find their way into 
and out of 
the monument--is an important feature for the author. He identifies the Vietnam 
War Memorial on the National Mall in Washington, D.C. as the forerunner of this 
trend in modern memorialization 
\parencite{britishassociationforholocauststudies2020}. The memorial was
revolutionary in many other ways, too. Young describes it as being a monument
to a war with few remaining supporters--one people might have wanted to forget. 
As such,
the memorial does not glorify its subject (unlike many of the other memorials
on the National Mall), but as Young explains in the talk, it is cut into the
landscape like a wound that will never heal; it thus presents a ``breach of
memorial code." This ``cut" or negative space is also used to signify the loss
the monument represents. Besides being cut into the landscape, the monument
only shows the names of the fallen of the Vietnam War on a slightly glossy
surface. This reflective surface puts the focus on the visitors because they
can see themselves in the memorial and thus become a part of the memorial 
\parencite{britishassociationforholocauststudies2020}. In Young's view, this
memorial influenced many following memorials, including the 9/11 memorial
discussed above and the \textcite{2020a}.\par

The arc of the memorial tradition starting with the Vietnam War memorial in
Washington, D.C., touching Holocaust memorials in Germany, and culminating in
the 9/11 memorial in New York City is a very interesting theory. I had never
made this connection or in all honesty given it any thought. This ``movement"
that Young describes is intriguing and something I am glad I learned a little
bit more about. I can say that I agree with Young's argument and his approach
which incorporates this movement into memorial analysis. 
By incorporating the memorialization process, the meaning memorials generate, and
the consequences of this meaning into the analysis of memorials, we can 
increase its depth. The recent focus on commemorating loss and not trying to
impose false memories or grandeur through memorials is a movement that I can
support; it fits with a more self-reflexive approach to memory and
memorialization.

\newpage

\section{References}

\noindent
British Association for Holocaust Studies. (2020, September 23). \textit{James
E. Young - The}\par 
\textit{Stages of Memory: Reflections on Memorialisation and Global
Commemoration}\par
[Video]. YouTube. URL \url{https://www.youtube.com/watch?v=6VEbNkt_DTs}\par

\noindent
Ehrenpreis, D. (2018). [Review of the book \textit{The Stages of Memory:
Reflections on Memorial}\par
\textit{Art, Loss, and the Spaces Between}, by J.E. Young]. Retrieved April 24,
2021, from\par
\url{http://caareviews.org/reviews/3497}\par

\noindent
\textit{James E. Young}. (n.d.). University of Massachusetts Amherst. Retrieved April 
22, 2021, from\par
\url{https://www.umass.edu/english/member/james-young}

\noindent
\textit{Memorial to the Murdered Jews of Europe}. (2020, December 7). Berlin.de. 
Retrieved April\par
22, 2021, from
\href{https://www.berlin.de/en/attractions-and-sights/3560249-3104052-memorial-to-the-murdered-jews-of-europe.en.html}{https://www.berlin.de/en/attractions-and-sights/3560249-3104052-}\par
\href{https://www.berlin.de/en/attractions-and-sights/3560249-3104052-memorial-to-the-murdered-jews-of-europe.en.html}{memorial-to-the-murdered-jews-of-europe.en.html}

\noindent
\textit{National September 11 Memorial \& Museum}. (n.d.). Retrieved April 24, 2021, from\par
\url{https://www.911memorial.org/}\par

\noindent
Young, J. E. (2008). The Texture of Memory: Holocaust Memorials in History. In
A. Erll \&\par
A. Nünning (Eds.), \textit{Cultural Memory Studies} (pp. 357–366). De Gruyter.

\end{document}


influenced german artists, remembering destruction with destruction; 

berlin memorial with negative space, perambulation, information center; new
dimension of the one in berlin because of what it tried to commemorate

- name Mahnmal vs Denkmal in german as solution to some of the questions that
he mentions

memorial lets people make up their own meaning

they don't try to tell the visitor what to think, because in the end the
visitors are the one that need to take action on their own terms; a memory or
memorial




if memorials do the memory work for us, we become more forgetful

stiffness of monuments is their downfall and their meaning is not at all
everlasting--it depends on the moment

process of creating the memorial is often as rewarding as the result

memory is contested

art history 

I tried really hard to come up with something that I disagree with here.
I watched a two hour lecture with the British Association for Holocaust
Studies, read a review of one of his exhibitions, a review of his latest book,
an article in the harvard design review. I read up on the memorials whose
creation he was involved in. Nothing that I came across really stuck me as
worthy of critique. Thus, I will simply add some information to the points that
he makes above. 

Points I can make:

- stolpersteine in his vein of monument


- 

\newpage

Structure

1. section on him and his achievements
2. summary of the text
3. specific memory studies aspects of his arguments
4. then discuss these specific things
    - holocaust memorial in Berlin that he was involved in
    - ground zero memorial
    - stolpersteine as example
    - german Mahnmal vs Denkmal, even though it is not consistently applied,
    one in vienna
    - purpose of monuments and statues
    - first reading on memorials
    - text by Rigney

Critique of JAMES E.YOUNG: The Texture of Memory: Holocaust Memorials in
History p. 357. 1000 words, typed, double-spaced. 

What does James E. Young argue in the work and what is your position regarding 
this argument? Explain why you agree or disagree with him.\par

Distinguished Professor Emeritus and Founding Director of the Institute for 
Holocaust, Genocide, and Memory Studies

Young's argument can be roughly split into 4 main sections

1. an overview of what memorials are; examples of hms in different countries
    - different memories of h, reasons for creating memorials
    - memorials are never one-sided
    - examples of polish hms; eu/isralei hms
    - case of germany:
        - difficulty in representation
        - what have artists done about it?
2. general problems with monuments
    - stiff, grand, ...
    - contested memory
3. art history and hms
    - why is this new view needed?
    - how does art critique need to be modified to make this work?
4. the influence of monuments
    - how do they interact with with the public at-large?
5. proper art critique
    - incorporate memory studies into critique of holocaust memorials
    - generation of meaning
    - role of the visitor
    - what is the purpose of this remembering?
\newpage


those issues using a fusion of art history and memory studies. 

He first introduces memorials as "". Examples of such memorials are given for
Poland, Israel, and Germany. In the case of Germany the author notes that
Germany's legacy as the successor state to the Third Reich puts it in
a difficult position--it is the perpetrator remembering its victims and trying
to incorporate this shame into something to identify with. 

Next, Young notes general problems he sees in memorials--they are too stiff and
grand.

He criticizes the current art-historical approach to hms because it is not
paying attention to the right things. He furthermore make suggestions as to
how art critique could live up to the expectations. 

In connection to that, the influence that monuments have or can have on the
public around them is discussed. 

Lastly, Young formulates his idea of a proper art-historical critique of
Holocaust memorials. It can be simplified to a proper incorporating of memory
studies into the art-historical analysis of Holocaust memorials. The needed
elements are an analysis of the generation of meaning, the role of the visitor
for the memorial,
% cite exhibit review
% cite first text on life of memorials
and an awareness towards which end is being remembered or what is the purpose
of this remembering.

In
short, Young argues that Holocaust memorials are shaped by their location,
their artist, and the artistic climate at their time of creation; they are
multi-faceted and evolving. The creation of a monument to crime, ugliness,
and horror (i.e. the Holocaust) poses a great challenge for two reasons: 
Firstly, it is a ``breach of memorial code" 
% cite the page
as the memorials do not depict triumph or
ideals and instead try to commemorate mass murder (this is especially
difficult for Germany, as it is a perpetrator remembering its victims).
Secondly, \textit{how} to depict mass murder in an appropriate and tactful way 
is an unanswered question. Some artists have made these questions and problems
themselves into pieces of art, while others have used negative form imagery and
shattered vessels to express the loss. 

- reason for being

458 words already

I think I agree with Young's argument here. He is a very accomplished scholar
and expert and the argument he is presenting here seems sound. It honestly
surprises me that he felt the needs to write this article, as most of the
points he is stressing seem to be reasonable assumptions that I would expect the
art history community to be aware of. 

The difficulty around the Holocaust memorials in Germany is understandable; one
point I would add to this is the distinction in German between the words
Denkmahl and Mahnmal. Denkmal is the proper translation of memorial and is used
for all types of memorials, including some Holocaust memorials.
% cite berlin memorial
On the other hand, there is the term Mahnmal, a variation of the word Denkmal.
Where Denk comes from Denken (to think) and gedenken (to remember), Mahnmal is
an admonishing, warning memorial. This creates a new category of memorial.
Examples include the holocaust memorial in vienna
% cite vienna holocaust memorial
One interesting point the author made is the fact that some people think the
didactic logic of memorials is too close to fascism to create memorials about
fascism. About this point I am not sure, looking at Umberto Eco's points of
fascism, we can see that memorials fulfill these roles (or that they do not).

incorporate first reading to support his argument about the importance of the
public in the work of a memorial.


