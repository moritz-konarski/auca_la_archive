\documentclass[12pt, a4paper, stu]{apa7}

\newcommand{\auth}{Moritz M. Konarski}
\newcommand{\authlast}{Konarski}
\newcommand{\titular}{Guilt and Responsibility in Remember and A Hidden Life}
\newcommand{\shorttitular}{Guilt and Responsibility}

\usepackage[no-math]{fontspec}
\setmainfont
[Path={/home/moritz/Documents/thesis.git/writing/drafts/00_fonts/}]
    {times.ttf}[
    BoldFont       = timesbd.ttf,
    ItalicFont     = timesi.ttf,
    BoldItalicFont = timesbi.ttf
]

\usepackage[american]{babel}
\usepackage{csquotes}
\usepackage[style=apa]{biblatex}
\addbibresource{../bib.bib}  

\hypersetup{
   colorlinks=true,
   allcolors=black,
   urlcolor=blue,
   urlbordercolor=blue,
   final,
   pdftitle={\titular{}},
   pdfauthor={\auth{}},
   pdfsubject={Memory Studies, Relection Paper}
}

\setlength{\parindent}{0.5in}

\title{\titular}
\author{\auth}
\shorttitle{\shorttitular}
\leftheader{\authlast}

\begin{document}

\noindent
Assignment 1 Option a)\\\vspace{-8pt}
\noindent
Name: \auth{}\\\vspace{-8pt}
\noindent
Date: \today{}\\\vspace{-8pt}
\noindent
Course: What in the Hell is Memory Studies? \\\vspace{-8pt}
\noindent
Instructors: Prof. Clyde R. Forsberg \& Hatice E. Mescioglu\\\vspace{-8pt}

\section{\titular}

\noindent
The movies \textit{Remember} \parencite{egoyan2015} and \textit{A Hidden Life} 
\parencite{malick2019} both tackle the Second World War, but they do it in very 
different ways. \textit{Remember} covers the memory and legacy of World War II 
through a story of revenge, while \textit{A Hidden Life} examines how those 
events interacted with a man and his convictions. To elaborate on this, first, 
the two movies will be briefly summarized. Then, they will be compared 
concerning the importance of guilt and responsibility for the main characters 
of both movies.\par

\textit{Remember} \parencite{egoyan2015} tells the story of 90-year-old 
Auschwitz survivor Zev Guttman. Despite having dementia and recently losing his 
wife, he, with the help of his friend Max, goes on a quest to find and kill the 
Nazi Blockführer Kurlander from Auschwitz who killed both of their families. 
When Zev finally finds him, he demands that Kurlander tell the truth about what 
he did. Kurlander reveals that he used to be Kunibert Sturm and that Zev used 
to be Otto Wallisch--both were Blockführer. Before being captured, they adopted 
false names and tattooed themselves with identification numbers. Enraged, Otto 
shoots and kills Kunibert and then himself. The final scene suggests that Max 
may have known of Otto's past and that he meant for the murder-suicide to 
happen.
The second movie, \textit{A Hidden Life} \parencite{malick2019}, is inspired by 
the life of Franz Jägerstätter, a modest Austrian farmer from rural St. 
Radegund. In 1940 he was drafted into the German Army and sent to basic 
training, but was let go after finishing. Before being called up again, he 
decides that he cannot serve in combat or swear an oath of allegiance to Adolf 
Hitler, even if it means being executed for it. His decision increasingly 
ostracizes and vilifies his family in the eyes of the other villagers. When he 
gets called up in 1943, he is arrested for refusing to swear allegiance to 
Hitler. He is brought to Berlin to stand trial, where he is eventually 
sentenced to death and executed. But not even the threat of death can deter him 
and he sticks to his convictions. This stands in contrast to many people he 
interacted with who indicated that they might not agree with what is happening 
but were unwilling to stand up for what they believe in. The judge who 
sentences Franz to death seems conflicted, maybe even guilty when faced with 
the decision.\par

A feeling of guilt is a driving force in both movies. The concept of guilt that 
will be used for this analysis is the one laid out in Karl Jaspers's 1946 "Die 
Schuldfrage" (The Question of Guilt) \parencite[see][p. 169]{langenohl2008}. 
He lays out four types of guilt \parencite[as summarized in][]{langenohl2008}:
\begin{itemize}
    \item criminal guilt: individual guilt that can be judicially sanctioned;
    \item political guilt: crimes committed on behalf of a political
        collective, all members of it are responsible;
    \item moral guilt: reaction to crimes against any human;
    \item metaphysical guilt: guilt relating to beings like gods.
\end{itemize}
These types of guilt can be used to analyze the protagonist's behavior in both
movies. In \textit{Remember}, Zev sets out to avenge wrong that has been done 
to his and Max's families--he is trying to punish the guilty. Kurlander's guilt
here is criminal guilt, as he was personally responsible for the deaths of many
people and Zev is trying to punish him for his crimes. Once he finds out that
Kurlander is actually Kunibert, Kunibert's guilt stays the same because he
still was a Blockführer at Auschwitz. Zev's newfound guilt suddenly puts him in
the same position as Kunibert. The rage and disbelief he feels are
understandable as he felt very strongly about the guilt of the man he was
hunting. This is obvious from the lengths he goes to to find him, trying even
after three failed attempts, smuggling a weapon across the border, etc. Zev was
driven by the criminal guilt of Kurlander to find him, and when he did, the
same guilt was put on him. It can be argued that, by killing Kurlander, he
punishes him for the crimes he committed (arguably also out of rage for
revealing Zev's true identity). When Zev kills himself, he could be punishing
himself for the same crime Kurlander committed, or because he cannot live with
the thought of having lived a lie. This murder-suicide may be exactly what Max
intended when he sent Zev on this mission--both Blockführer who killed his
family are now dead. If one accepts that Max orchestrated Zev's campaign, he
punished both criminals for the crimes they committed. In this case, too, guilt
is the driving factor, as Max is driven by a desire to get revenge for the acts
that made Zev and Kurlander guilty.\par

Guilt is also an important motivator for Franz in \textit{A Hidden Life}. The 
idea of criminal guilt does not apply to him as it does to Zev and Kurlander.
Rather, he seems to be worried about becoming politically (or even criminally)
guilty by participating in the Nazi war effort or any of their programs. By not
participating, he is saving himself from that type of guilt. Moral guilt is
also a factor for him; he seems to feel for the countries and people that Nazi
Germany invaded and he wants no part in that. Most of all, metaphysical guilt
moves him. The movie makes it very clear how deeply religious he is and he
himself says that he could not participate in Nazi activity while keeping
a clean conscience before God. Franz is so bent on not feeling guilty that he
is willing to sacrifice his life to avoid it. Thus, guilt is the driving force
in both \textit{Remember} (in bringing Kurlander to justice for both Zev and
Max) and in \textit{A Hidden Life} (in Franz paying the ultimate price to not
become guilty) and it can be used to explain the characters' actions.\par

The concept of guilt is closely related to the concept of responsibility, which
is also important in both movies. According to \textcite{talbert2019}, to be
responsible for something, one must have a certain power over one's actions,
the ability to do something else. One approach to responsibility that is
helpful in analyzing these movies is the forward-looking approach that holds
that responsibility is influenced by the possible rewards of an action
\parencite{talbert2019}. Applying this idea to Zev's situation when he fled
Auschwitz, we observe that there were no possible rewards for being responsible
(taking responsibility for what he did). If the approaching Soviet Army
captured him, he might have been executed on the spot. Even after reaching the
US, there were no benefits in admitting what he did and taking responsibility
for it, so he never did. When he committed suicide he again evaded
responsibility by not even facing what he did. Zev also always had the option
to do something else: in Auschwitz he could have resisted (or even just have
been less cruel), and when he fled he could have turned himself in. The case of
Zev is complicated by the fact that he has dementia, as it can be argued that
he has diminished agency and thus less to no power over his actions. During his
manhunt, he is guided by the letter Max wrote him that lays out what he has to
do. If Max is pulling the strings and controlling Zev's actions, Zev cannot be
held responsible for them. This means that the crimes committed during the
manhunt are not Zev's fault, but it is Max who is guilty of them. The murder of
Kurlander and (because the encounter only happened as a result of Max's
actions) the murder of the Nazi State Trooper make Max criminally guilty. It is
the same type of guilt as the man that he sent Zev to kill (this is not meant
to equate Max's actions to Nazi mass murder, it is simply the same type of
guilt). The movie does not show what happens after the murder-suicide so we do
not know if Max takes responsibility for what he did. While admitting to
a murder is not an easy or common thing, he may share what he did because he
obviously hated both Zev and Kurlander for what they did and he might want
people to know that he avenged his family and took responsibility.\par

For Franz, responsibility determines his actions. He is feeling both morally
obligated to not contribute to the Nazi apparatus, but first and foremost he
feels a responsibility towards God to do what is right. He takes this
responsibility so seriously that he sacrifices his life for it. He also always
had options, he could have gone along with the Nazi program and been completely
fine. As his execution draws near, he is given dozens of opportunities to
change his mind and swear the oath to Hitler, but he does not. Even when he is
intimidated, he refuses to change his mind. One argument against Franz being
responsible is that he is willing to have his whole family ostracized and
vilified because of his personal struggle, but this is outside the scope of
this paper. When it comes to responsibility, Zev and Franz are polar opposites.
Zev never takes responsibility for any of his actions, even in death, while
Franz takes his responsibility to God so seriously that he is willing to die
for it.\par

Guilt and responsibility are the main motifs of \textit{Remember} and \textit{A
Hidden Life}, but their portrayal of them is very different. Zev is motivated
by guilt but never takes any responsibility for his own actions. Max uses Zev
to carry out his revenge on the men who are guilty of killing his family,
whether he takes responsibility for that is unknown. Franz on the other hand is
so committed to staying without guilt and being responsible that he sacrifices
his life and his family's well-being for it. In \textit{A Hidden Life} the
sentiment towards the protagonist is positive as we see Franz prevail in the
face of abuse, degradation, and even death; he becomes a martyr for what he
believes in. \textit{Remember's} sentiment towards the characters is more
negative and the characters are less selfless and more violent. Zev has more
blemishes than Franz (he murders and turns out to be the villain) and Max uses
a demented man to carry out his revenge and maybe even plans his death. After
presenting us with these characters, \textit{Remember} leaves it up to us to
judge their actions. While Franz in \textit{A Hidden Life} is the hero of the
story (even if his family suffers because of his actions), Zev and Max can be
seen as heroes or villains because in addition to their negative traits
mentioned above, both of them have very strong and understandable motivations
that drive them (murder is bad, but if someone brutally murders one's family,
retaliation can be understandable). These different portrayals nonetheless
carry the same message. They both raise important questions about taking
responsibility for one's actions and about the influence of guilt on human
decisions without necessarily resolving the issue. The characters of both
movies act out of the conviction that what they are doing is right, they
believe they have to do something because it is their responsibility. Max has
a responsibility toward his dead family, Zev toward Max, and Franz toward his
conscience and God. They are all willing to suffer in order to take
responsibility.\par

I agree with what these two movies say about doing what is right.
I admire people who do what they think is right no matter what it costs them
\parencite[see for example][]{whiterose2021}. Additionally I appreciate that
the negative sides of standing up for your beliefs is shown in these movies.
Compared to \textit{A Hidden Life}, I prefer the representation of the
characters in \textit{Remember} because even though they are more flawed,
they are not represented in an almost savior-like fashion, as Franz is.
\textit{A Hidden Life}, in trying to make Franz's fall seem more dramatic and
to make him a good guy, makes him look too-perfect. He is a loving father who
adores his kids and wife, he is a devout Christian who helps repair and
maintain the church, he stands up for what he believes in but is non-violent
while doing so, he even turns the other cheek, and remains calm until his
death. Even if this is what the real Franz did, as a character it is not
convincing in my opinion. Regardless of that, both movies are worth watching
and they raise important questions in connection to guilt and responsibility.

\newpage

\section{References}

\noindent
Egoyan, A. (Director). (2015). \textit{Remember} [Film]. Serendipity Point
Films.\par
\noindent
Langenohl, A. (2008). Memory in Post-Authoritarian Societies. In A. Erll \& A.
Nünning\par
(Eds.), \textit{Cultural Memory Studies} (pp. 163–172). De Gruyter.\par
\noindent
Malick, T. (Director). (2019). \textit{A hidden life} [Film]. Fox Searchlight
Pictures.\par
\noindent
Talbert, M. (2019). Moral Responsibility. In E. N. Zalta (Ed.), \textit{The Stanford Encyclopedia of} \par
\textit{Philosophy} (Winter 2019). Metaphysics Research Lab, Stanford
University. Retrieved\par
March 23, 2021, from\par
\url{https://plato.stanford.edu/archives/win2019/entries/moral-responsibility/}\par
\noindent
White Rose. (2021). \textit{Wikipedia}. Retrieved March 23, 2021, from\par
\url{https://en.wikipedia.org/w/index.php?title=White_Rose&oldid=1012459760}\par
Page Version ID: 1012459760

\end{document}
