\documentclass[../00_main.tex]{subfiles}

\begin{document}

\chapter{Class 24.09.2020}

\section{The nominative case -- Атооч Жөндөмө}

\begin{itemize}
    \item used to answer the questions "Ким?" (Who?) and "Эмне?" (What?)
    \item with "Бул(ар)" (this) or "Тиги(лер)" (that)
\end{itemize}

\subsection{Questions}

\begin{center}
\begin{tabular}{| l | l |}\hline
    Бул ким?    & Who is this?      \\\hline
    Бул эмне?   & What is this?     \\\hline
    Тиги ким?   & Who is that?      \\\hline
    Тиги эмне?  & What is that?     \\\hline
\end{tabular}
\end{center}

\subsection{Answers}

\begin{center}
\begin{tabular}{| l | l |}\hline
    Бул мугалим.    & This is a teacher.    \\\hline
    Бул китеп.      & This is a book.       \\\hline
    Тиги бала.      & That is a child.      \\\hline
    Тиги үйлөр.     & That is a house.      \\\hline
\end{tabular}
\end{center}

\subsection{Postpositions}

\begin{center}
\begin{tabular}{| l | l |}\hline
    Жөнүндө     & about             \\\hline
    Тууралуу    & about             \\\hline
    Менен       & by, with, and     \\\hline
    Үчүн        & for, to           \\\hline
    Cыяктуу     & like, as          \\\hline
    Аркылуу     & through           \\\hline
    Боюнча      & according to      \\\hline
\end{tabular}
\end{center}

\subsubsection{Менен with pronouns}

\begin{itemize}
    \item Биз, Сиз, Силер, Сиздер, Алар -- менен
    \item Мени, Сени, (ал)--Аны, (бул)--Муну -- менен
\end{itemize}

\subsubsection{Examples}

\begin{center}
\begin{tabular}{| l | l |}\hline
    Автобус менен барам     & I go by bus               \\\hline
    Досум менен барам       & I go with my friend       \\\hline
    Сен жөнүндө             & about you                 \\\hline
    Сиз үчүн                & for you                   \\\hline
    Биз сыяктуу             & like us                   \\\hline
    Сиз аркылуу             & through you               \\\hline
    Китеп боюнча            & according to the book     \\\hline
\end{tabular}
\end{center}

\end{document}
