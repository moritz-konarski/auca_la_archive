\documentclass[../00_main.tex]{subfiles}

\begin{document}

\chapter{Class 17.09.2020}

\begin{itemize}
    \item epics are very rich, mostly oral
    \item until the Russians came, most people were illiterate
    \item oral lit -> akyn's poetry -> soviet period -> post soviet literature
\end{itemize}

\section{Epics}

\begin{itemize}
    \item read the definition of an epic
    \item definition: long narrative poem in elevated style; characters of high
        position in adventures forming an organic whole with the central figure;
        episodes are important to the history of a nation or race
    \item other examples: Song of the Niebelung's
    \item Manas is an important part of Kyrgyz history and culture -- he is 
        everywhere
    \item Manas is heroic of course, it may contain some historical elements, 
        like wars between KG and Uyghurs, or maybe Chinese or other historical 
        events
    \item it is important because it sends a message of unity
    \item the descriptions and depictions of Manas can tell us something about 
        what an epic poem is
\end{itemize}

\section{Main characteristics of the epic}

\begin{itemize}
    \item begins \textit{in medias res} (middle of the plot)
    \item- vast setting -- many countries, nations, world, even universe
    \item starts with invocation to a muse (epic invocation), most western 
        epics -- Sacred world of Manas
    \item statement of theme at start
    \item use of epithets -- repeated phrases that help remember the poem
    \item long lists of things -- epic catalogue
    \item long formal speeches -- common in all epics
    \item includes divine interventions on human affairs -- Manas has spirits 
        and rituals and powers that influence the story
    \item heroes embody the values of the epic's civilization -- unity in 
        Manas' case "which was broken has been restored"
\end{itemize}

\section{Manaschys}

\begin{itemize}
    \item you kill nine animals
    \item recite for a while to become a teller
    \item today manaschys come together to recite one after the other for days 
        upon days as a way to honor it
\end{itemize}

\section{Ready-made elements}

How can they memorize so much text? they all have the same elements
\begin{itemize}
    \item birth of a hero
    \item growing-up of a hero
    \item praise of weapons
    \item preparations for battle
    \item din of the battle
    \item altercations prior to battle
    \item descriptions of persons and steeds
    \item characteristics of famous heroes
    \item praise of a bride's beauty
    \item description of dwelling, yurt, feast
    \item invitations to feast
    \item death of a hero
    \item description of landscape, nightfall, daybreak
\end{itemize}

\section{Structure of Manas}

\begin{itemize}
    \item brith of Manas
    \item childhood
    \item becoming khan
    \item fights
    \item meeting Almambet
    \item marriage to Kanykei
    \item Kokotoi's memorial
    \item great battle
    \item death
\end{itemize}

\section{History}

\begin{itemize}
    \item Chokan Valikhanov (1835-1865)
        \begin{itemize}
            \item Russian Imperial officer, agent, amateur ethnologist
            \item in 1856 in Issyk Kul recorded 2 days of "Kokotoidin ashy" 
                from Nazar Bolot -- about 3251 lines about brawls, feasting, 
                horse racing, fames
            \item held in archives of Oriental Archives, St. Petersburg
            \item 1861 short publication
            \item 1904 Russian translation
            \item then Hatto translation
        \end{itemize}
    \item Radloff transcription -- 1862, 1869
        \begin{itemize}
            \item many different sections
            \item published 1885 in St. Petersburg
        \end{itemize}
    \item these old version are somewhat nice because they are original and not
        too corrected -- probably a good thing in the eyes of the scholars
\end{itemize}

\section{Myths about Manas (by A. Wachtel)}

\begin{itemize}
    \item not unique etc
    \item improve and edit the dialogues
\end{itemize}

\end{document}
